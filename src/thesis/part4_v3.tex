\thispagestyle{fancy2}

提案手法のパイプラインは大きく分けて,(1)物体検出,(2)ReIDによる観測系列の取得,そして(3)カルマンフィルタによる状態推定の三つに分けられ,提案部分はこの(2)と(3)であった.よって評価のために理想的な物体検出の結果を生成するデモデータの作成を行う.よって第四部ではまず,デモデータの作成方法をはじめに述べ(\ref{sec:demodata}),そのあとにデモデータに対して提案部分であるReIDと状態推定の評価を行う(\ref{sec:pipeline}).また本研究においてスライス数の設定は結果に大きな影響を与えるハイパーパラメータである.よってこのスライス数を変化した場合の評価を次に行う(\ref{sec:number_of_slice}).そして最後に,物体検出を考慮した評価を行う.この物体検出の評価では,実際の対象である共焦点顕微鏡から得られた細胞のデータを適応し,どのような結果が得られるかを評価する(\ref{sec:cell_detection}).そして最後の章で,この検出の結果が全体のパイプラインにどのような影響を与えるかを評価し,またその対処方法について議論する(\ref{sec:covering_detection}).

\section{デモデータの生成方法}
\label{sec:demodata}

本章ではデモデータの作成手順を順に述べる.基本的な数式は\ref{sec:setting}で説明した問題設定の数式を再掲している.また本論文では,対象空間の広さを常に$\ell = 100$に固定している.またこの章では\ref{sec:setting}で用いた離散時刻と連続時刻を用いるため,再度$t$を連続時刻,$\tau$を離散時刻として扱う.

\begin{enumerate}[label=手順\arabic*]
    \item 連続的な時刻$t=0$において設定した物体数$n_b$個の物体のランダム生成を行う.物体の状態変数は式\ref{eq:continous_state_vecotr}で表される位置や,それらを微分した速度,加速度で表される.よってデモデータを生成する際にはこれらの値を決める必要がある.本研究では位置は$[0, \ell]$,速度はハイパーパラメータ$v_{\text{range}}$を用いた$[-v_{\text{range}}, v_{\text{range}}]$,半径はハイパーパラメータ$r_{\text{min}}, r_{\text{max}}$を用いて$[r_{\text{min}}, r_{\text{max}}]$から一様サンプリングして初期化する.また加速度は$0$で初期化する.
    \item 次に,連続的な物体の挙動を再現するために,$dt=0.01$など小さな値を用いて位置,速度,加速度を更新する.このときの更新式は式\ref{eq:sort_transition}と同様に,
    \begin{equation}
        \label{eq:transition}
        \begin{aligned}
            x_{t+1} &= x_t + \dot{x}_t dt + \frac{1}{2} a_t dt^2
            \\v_{t+1} &= \dot{x}_t + a_t dt
            \\a_t &\sim \mathcal{N}(0, A_x)
        \end{aligned}
    \end{equation}
    で計算される.すなわち物体に加えられる外力は正規的なランダム力だとする.ここで冗長性のため$x$に対しての式のみ記している.また加速度に加えられるノイズの分散は$x, y, z$ですべて共通とし,ハイパーパラメータ$\sigma_{\text{noise}}$として設定する.
    \item 共焦点顕微鏡が一枚の二次元画像を取得するのにかかる時間が$t_c$である場合,連続的な時刻が$t_c$の整数倍になるタイミングで画像の取得を行う.この$t / t_c$は\ref{sec:setting}で述べた離散時刻$\tau$に相当する.この画像は\ref{sec:setting}で説明した式\ref{eq:appearant_radius}や式\ref{eq:circle_region}に応じて生成する.またこのときのスライス$s$の値は,離散時刻$\tau$をさらに$n_s$で割った余りである.これにより,$n_s$枚取得するまでは取得のたびにスライスの高さが式\ref{eq:delta_time_and_z}における$\delta z$ずつ高くなり,$n_s$枚に達したところで$s = 0$に戻る.
    \item 手順3の画像取得と同時に,評価に必要な正解ラベルも生成する.まずReIDと状態推定の評価のために検出バウンディングボックスの正解ラベルを生成する.このバウンディングボックスの$(x, y)$には,各物体の連続的な位置$x(t), y(t)$を採用し,$(w, h)$には式\ref{eq:appearant_radius}で示した見かけ上の半径$r_{\text{app}}(t)$の二倍の値を用いる.また状態推定の評価のために,各物体について一つの$\tau$につき一つの状態変数を生成する.状態推定で推定する状態変数は$s = 0$の画像が取得された時刻での状態を示すため,$s = 0$の画像が取得されるタイミングで各物体の位置を出力する.
    \item デモデータの生成ではランダムな加速度を与えているため,物体が対象領域$\Omega$を出てしまうことも考えられる.すると速度範囲を変えた際などに平等な評価を行うことが困難になるため,領域を出てしまった時点でその分の新しい物体の生成を行う.この生成方法は手順1と同様である.物体のすべての領域が対象領域$\Omega$と共通部分を持たなくなったら領域を出たと判断する.
\end{enumerate}

以降の実験では特記しない限り,$n_b = 10$,$v_{\text{range}} = 0.1$,$r_{\text{min}} = 5$ ,$r_{\text{max}} = 10$,$\sigma_{\text{noise}} = 0.01$,$t_c = 1$,$n_s = 101$を基準のデモデータとして使用する.

\section{ReIDと状態推定のパイプライン評価}
\label{sec:pipeline}

    本章では,共焦点顕微鏡から得られた画像データを物体検出にかけて得られたバウンディングボックス群を入力とし,各物体の三次元情報を推定,予測するReID+状態推定のパイプライン評価を行う.今回はデモデータを用いることで,FPとFNが存在しない理想的な物体検出が行われたとして上記の評価を行う.またここでいう状態推定とは,時刻$t-1$までの物体検出情報が得られたときの,時刻$t$の状態の推定のことを指す.

    また続く節では,ReIDと状態推定を別々にして評価を行う.ReID手法の評価では,\ref{sec:proposed_method}章で提案した四つの手法の評価および議論を行う.また状態推定の評価では,SliceKalmanFilterの詳細な評価と直感的な手法との比較を行う.

    \subsection{パイプライン評価}
    \label{subsec:pipeline_evaluation}

        \subsubsection{実装詳細}
        パイプライン評価では表\ref{tab:reidentification_methods}に示す四つの手法を評価する.時間方向の追跡を打ち切る閾値は$\sigma_{\text{sort-act}} = 3$,空間方向の追跡を打ち切る閾値は$\sigma_{\text{d-sort-act}} = 1$とした.またすべてのカルマンフィルタにおいて,システムノイズ分散は$100$,
        観測ノイズの大きさは$1.0$,初期時刻における推定状態分散は$\Sigma_0 = 100 * I_n$とする.ここでの$n$は状態変数の長さであり$I_n$は$n$次の単位行列である.また動作環境は,OSがUnuntu 20.04.6 LTS(Focal Fossa)であり,CPUがIntel(R) Xeon(R) Gold 6338 CPU @ 2.00GHzの32コアである.

        \subsubsection{比較手法}
        本実験ではReIDに\ref{sec:existing_method}で説明したSORTと外観情報を用いずに高精度なReIDを達成するByteTrack\cite{zhang2022bytetrack}を適用したものを比較手法として評価する.これらの比較手法は単一のスライスにしか適用できないため,各スライスで追跡される情報を独立したものとしてReIDの結果を得る.またこれらの比較手法では,単純な状態推定を行う.すなわち,単一のスライス$s$におけるバウンディングボックスの軌跡から予測された時刻$t + 1$のバウンディングボックスが,$[x, y, w, h]$で表されるとき,時刻$t+1$の状態推定値$\bm{s}_{t+1}^{\text{pred}}$を,
        \begin{equation}
            \label{eq:naive_state_estimation}
            \bm{s}^{\text{pred}}_{t+1} = \left[x,~ 0,~ y,~ 0,~ s \delta z,~ 0,~ \frac{\max(w, h)}{2}\right]
        \end{equation}
        として得る.すなわち物体の中心がそのスライス上かつバウンディングボックスの中心に一致し,そのバウンディングボックスが球体の半径を用いて表せられる円に外接しているとして推定している.
        
        \subsubsection{評価指標}
        
        ReIDの状態推定を組み合わせたパイプラインの出力は,時刻$t$において追跡している各物体$i \in \mathcal{I}_t$の時刻$t+1$の状態推定$\bm{s}_{t+1}$である.この状態推定値の推定精度を評価するために,三次元空間における予測球体と正解ラベル球体の共通領域度合いを示すIoUの平均値(以後meanIoUと呼ぶ)を用いる.ある二つの球体$\bm{s}_1 = [x_1, y_1, z_1, r_1]$と$\bm{s}_2 = [x_2, y_2, z_2, r_2]$が存在するとき,IoU$(\bm{s}_1, \bm{s}_2)$は,
        \begin{equation}
            \label{eq:sphere_iou}
            \begin{aligned}
                d &= \sqrt{(x_1 - x_2)^2 + (y_1 - y_2)^2 + (z_1 - z_2)^2}
                \\V_{\text{inter}} &= \left\{
                    \begin{aligned}
                        &\frac{\pi}{12 d} \left\{d - (r_1 + r_2)\right\}^2 \left\{d^2 + 2(r_1 + r_2)d - 3 (r_1 - r_2)^2\right\},& \quad &\textbf{if } d < r_1 + r_2
                        \\ & 0,& &\textbf{else} 
                    \end{aligned}
                \right.
                \\\text{IoU}(\bm{s}_1, \bm{s}_2) &= \frac{V_{\text{inter}}}{\frac{4}{3}\pi r_1^2 + \frac{4}{3}\pi r_2^2 - V_{\text{inter}}}
            \end{aligned}
        \end{equation}
        で表される.

        meanIoUは,予測された球体ごとに正解ラベル球体とIoUを計算し,これらの平均値を取る.また評価の際には,予測されたある球体が正解ラベルの球体におけるどの球体に対応する予測なのかを知ることができない.よって評価の際にはスコアであるmeanIoUが最大になるようにこの対応付けを行う.すなわち$f_{\text{sim}}$をIoUとしたアルゴリズム\ref{alg:hungarian_matching}を利用する.

        \subsubsection{結果}

        \begin{figure}[t]
            \begin{subfigure}[b]{0.5\linewidth}
                \centering
                \includegraphics[width=\linewidth]{fig/meanIoUs.pdf}
                % \caption[パイプライン評価におけるmeanIoU比較]{パイプライン評価におけるmeanIoU比較:提案手法の詳細については表\ref{tab:reidentification_methods}を参照されたい.}
                \label{fig:meanIoU_pipeline}
            \end{subfigure}
            \hfill
            \begin{subfigure}[b]{0.5\linewidth}
                \centering
                \includegraphics[width=\linewidth]{fig/prediction_count.pdf} 
                \label{fig:prediction_count}
            \end{subfigure}
            \vspace{-3zh}
            \caption[パイプラインにおける既存手法との比較評価]{パイプラインにおける既存手法との比較評価:実線が提案手法,点線が既存手法による結果を示す.(左)meanIoUは予測球体と正解ラベル球体の重なり具合を示し,最大が$1.0$である.(右)縦軸が予測された追跡対象の個数を表す.既存手法ではスライスで分割された分かなり冗長な予測がされている.}
            \label{fig:pipeline_assessment}
        \end{figure}
        
        図\ref{fig:pipeline_assessment}に結果を示すように,すべての時刻において四つすべての提案手法が既存手法の精度を上回っている.この結果では予測された球体のうち,IoU$ > 0$のもののみを換算しているため,予測の精度が良いものに限定しても提案手法の予測精度が大きく既存手法を上回っていると言える.また図\ref{fig:pipeline_assessment}(右)では,既存手法の予測数が提案手法に比較してかなり冗長であることを示している.これはスライスごとに予測を行っているためと考えられる.これは実際に追跡による予測を参考にする際に,どれを選択すればよいか判断しづらいという問題に発展してしまう.一方で提案手法では$n_b$である$10$付近に予測数を抑えられていることが分かる.

    \subsection{状態推定手法単体の評価}
    \label{subsec:evaluate_estimation}

    図\ref{fig:pipeline_assessment}では,四つの提案手法にも精度の違いが確認された.そこでここからの節では,ReIDと状態推定を分けて四つの提案手法の違いを検証する.また本節では,パイプラインの順番と前後するが,議論の容易さのために下流側の状態推定から評価を行う.

    本節ではSliceKalmanFilterの精度を確かめるために,検証用データの作成手順4(\ref{sec:demodata}参照)において作成した,ID付きのバウンディングボックスを入力としてそれぞれのIDにおける状態推定を行う.すなわち理想的なReIDが行われた場合のmeanIoUを確認する.またSliceKalmanFilterの詳細な評価を行うため,カルマンフィルタにおける事後分布$p(\bm{s}_t \mid \bm{o}_1, \dots, \bm{o}_t)$の期待値と共分散の対角行列成分を正解ラベルの値と比較する.

        \subsubsection{結果}

        \begin{figure}[t]
            \begin{subfigure}[b]{\linewidth}
                \centering
                \includegraphics[width=0.55\linewidth]{fig/meanIoUs_withGT.pdf}
                \caption[理想的なReIDを用いた場合のmeanIoU]{理想的なReIDを用いた場合のmeanIoU:紫色の実線が,正解ラベルを用いた状態推定の精度を表す.}
                \label{fig:meanIoU_withGT}
            \end{subfigure}
            \\
            \begin{subfigure}[b]{\linewidth}
                \centering
                \includegraphics[width=\linewidth]{fig/GTperformances/003.pdf}
                \caption[SKFの事後分布と正解ラベルの比較]{SKFの事後分布と正解ラベルの比較:左上から順に,$x$,$y$,$z$,$r$,$\dot{x}$,$\dot{y}$,$\dot{z}$の正解ラベル(黒実線),事後分布期待値(青実線),事後分布の$95\%$信頼区間を表す.}
                \label{fig:SKF_performance}
            \end{subfigure}
            \caption[理想的なReIDを用いたSKFの状態推定精度評価]{理想的なReIDを用いたSKFの状態推定精度評価}
        \end{figure}

        まず理想的なReIDを行った場合のmeanIoUの結果を図\ref{fig:meanIoU_withGT}に示す.図に示す通り,提案手法の四つによってReIDされた観測系列からでも,理想的なReIDに近い精度を推定できていることが分かる.また図の一部でSliceKalmanFilterを時間方向の対応付けに用いた場合に理想的なReIDを用いている場合よりも精度が高い理由については後に議論する.

        また図\ref{fig:SKF_performance}は,理想的なIDを用いた際の,ある一つのIDに対して正解ラベル,事後分布の平均値,そしてその分散共分散行列の対角成分を示している.図において$y$のグラフに着目すると,物体が対象領域の端($\ell = 100$)付近に近づくにつれて精度が落ちていることが分かる.またそのタイミングと同時に,ほかの状態変数の精度も落ちている.これは物体が対象領域の端に近づくにつれて検出されるバウンディングボックスが欠け,推定がうまくされていないことが容易に想像できる.このような傾向が他の物体に対しても見られ,反対にすべての時刻で大きく推定がずれているものも確認されなかった.すなわち図\ref{fig:meanIoU_withGT}において$\text{meanIoU} = 1.0$より劣る原因は,SliceKalmanFilterが,画像から物体が見切れる状況を扱いきれていないためだと考えられる.

    \subsection{ReID手法単体の評価}
    \label{subsec:evaluate_reidentification}

    図\ref{fig:meanIoU_withGT}に示したように,時間方向の対応付けにSliceKalmanFilterを用いた場合は理想的なReIDと同様かそれ以上の精度を示し,反対に各スライスのSORTを用いた場合には精度が下がることが確認された.そこで本節では.MOTにしばしば用いられる評価指標と新たに導入するDetection-Mapという図示方法を利用して,これらの違いの議論を行う.

    \subsubsection{MOTの評価指標}

    MOTにおける潜在的なエラーは,検出の可否,IDの振り違い,そして検出位置のズレの三つから構成される.本節では理想的な物体検出を想定しているため,この中で本節に用いることが有効であるのはIDの振り違いを評価するものである.よって本節では,ID-switch(以後IDSWと呼ぶ)とAssociation-Accuracy(以後AssAと呼ぶ)という評価指標を用いる.

    MOTの評価指標の説明を行うために,いくつか用語の導入を行う(ただしこれらの用語はMOTの評価指標の一つであるHOTAに関する論文\cite{luiten2021hota}に揃えている).まず検出されたバウンディングボックスをprDets,バウンディングボックスの正解ラベルをgtDetsという集合として表す.またこれらには,どの個体由来であるかを示すprIDsとgtIDsの集合に含まれるIDが割り振られている.すなわちある検出バウンディングボックスprDetには,prID(prDet)が割り振られている.また評価指標を算出する際には,どのprIDがどのgtIDに対応するのか(IDの対応付け)やどのprDetがgtDetに対応するのか(Detの対応付け)などが必要であり,これらは基本的に最終的なスコアを最大化するように対応付けられる.

    IDSW\cite{luiten2021hota}は,あるgtIDを持つgtDetsに着目したとき,そのgtDetsに対応付けられたprDetsの中にいくつのprIDが混在しているか(実際にはswitch回数を計算するので個数$-1$)を数え上げたものである.もし理想的なReIDが行われた場合,この値は$0$になる.

    またAssA\cite{luiten2021hota}は,前述の三つのエラーをすべて考慮したHOTAを構成する評価指標のうち,当初の目的であるIDの振り違いを評価できる評価指標である.このAssAは,ある対応付けが行われたprDetとgtDetがあるときに,それぞれが持つIDが振られているprDetsとgtDetsに着目し,どれだけその集合が一致しているかを一般的なPrecisionやRecallの考えに則って算出したものを平均したものである.もし理想的なReIDが行われた場合,この値は$1.0$になる.

    \subsubsection{Detection-Mapによる可視化}

    また本節では詳細な議論を行うために,Detection-Mapという図示によってReIDの結果の可視化を行う.このDetection-Mapでは,追跡を行った各IDの物体に対して,そのIDを振られたバウンディングボックスがどの(時刻$t$とスライス$s$で特徴づけられる)画像で検出されたものであるのかを示す.すなわち,各物体がどの画像で検出されたのか,どれだけその物体を追跡できているのかを示すことができる.
    
    \subsubsection{結果}

    \begin{table}[t]
        \centering
        \caption[ReIDにおける提案手法の比較]{ReIDにおける提案手法の比較:太字は最良値を示している.\# Predictionsは,予測されたprIDsの個数を表している.}
        \label{tab:metrics_reidentification}
        \begin{tabular}{l|ccc}
            Methods & \# Predictions & IDSW & AssA 
            \\\hline \hline
            only SORT & $49$ & $324$ & $32.2\%$
            \\ only SKF & $35$ & $\bm{18}$ & $93.8\%$
            \\ SORT + DepthSORT & $\bm{33}$ & $23$ & $\bm{95.4\%}$
            \\ SKF + DepthSORT & $52$ & $25$ & $94.3\%$
        \end{tabular}
    \end{table}

    \begin{figure}[t]
        \centering
        \includegraphics[width=\linewidth]{fig/detection_maps.pdf}
        \caption[各ReID手法によるDetection-Mapの一部]{各ReID手法によるDetection-Mapの一部:(a)理想的なReIDに近いDetection-Map(これはSORT+DepthSORTによる結果の一部である).(b)SORT+DepthSORTにおいて,IDSWが生じているDetection-Map.(c)SKFのみによるReIDにおけるDetection-Mapの一部,図のように少し欠けている部分が見られる.(d)SORTのみによるReIDにおけるDetection-Mapの一部.スライス方向への対応付けができていないことが分かる.}
        \label{fig:detection_maps}
    \end{figure}

    表\ref{tab:metrics_reidentification}および図\ref{fig:detection_maps}に結果を示す.まず時間方向の対応付けのみを行い,かつその対応付けをSORTのみで行った場合(only SORT),図(d)に示すように他のスライスにおけるバウンディングボックスに対応付けることができず,表に示すようにMOT自体の精度が下がってしまっている.

    一方評価指標で大きな差が見られなかった他の三つでは,図(a)と図(c)を比較することで,DepthSORTを用いない場合はある程度の対応付けの見逃しが発生していることが分かる.しかしながら一方でSORTで対応付けしたあとにDepthSORTを利用すると,表\ref{tab:metrics_reidentification}のIDSWと図(b)に示すように他の個体由来のバウンディングボックスに同じIDを振ってしまっていることが確認される.おそらくこれは,各スライスにおけるバウンディングボックス予測が他の個体由来の検出バウンディングボックスと位置が近く,間違えて対応付けられてしまったためだと考えられる.

    すなわち対応付けにDepthSORTを実施しない場合は,見逃しが発生するが状態推定に大きな影響を及ぼさない.反対にDetphSORTを実施していても,SORTなどにより他の個体由来のバウンディングボックスが対応付けられてしまうと,状態推定に大きな影響を及ぼし,図\ref{fig:pipeline_assessment}に示すような精度低下につながったと結論付けられる.

\section{スライス数の変化に対する評価}
\label{sec:number_of_slice}

\section{細胞に対する物体検出の評価}
\label{sec:cell_detection}

\section{物体検出のFPとFNが与える効果および対処}
\label{sec:covering_detection}
