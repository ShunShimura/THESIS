\thispagestyle{fancy2}

共焦点顕微鏡から得られるデータで複数物体追跡を行う場合のパイプラインは大きく分けて,(1)物体検出,(2)ReIDによる観測系列の取得,そして(3)カルマンフィルタによる状態推定の三つに分けられ,提案部分はこの(2)と(3)であった.そのため理想的な物体検出を想定し,検証用データを用いて提案部分の評価を行う.よって本章では,まずはじめに検証用データの作成方法をはじめに述べ(\ref{sec:demodata}節),提案部分のパイプライン評価を行う(\ref{sec:pipeline_evaluation}節).そして詳細な評価のために,続く節ではReIDと状態推定を段階的に評価する(\ref{sec:stage_evaluation}節).また本研究において,対象空間をどれだけ細かくスライスするかを決めるスライス数は,重要なハイパーパラメータである.よってこのスライス数を変化させた場合の影響を評価する(\ref{sec:number_of_slice_effect}節).また最後の節では,実際のプロトプラスト化した細胞を対象として物体検出の評価を行い,この精度が後続のReIDおよび状態推定に及ぼす影響および対処方法を議論する(\ref{sec:pipleline_evaluation_with_detection}節).

\section{検証用データの作成方法}
\label{sec:demodata}

本章では検証用データの作成手順を順に述べる.基本的な数式は\ref{sec:setting}で説明した問題設定の数式を再掲している.また本論文では,対象空間の広さを常に$\ell = 100$に固定している.またこの章では\ref{sec:setting}で用いた離散時刻と連続時刻を用いるため,再度$t$を連続時刻,$\tau$を離散時刻として扱う.

\begin{enumerate}[label=手順\arabic*]
    \item 連続的な時刻$t=0$において設定した物体数$n_b$個の物体のランダム生成を行う.物体の状態変数は式\ref{eq:continous_state_vecotr}で表される位置や,それらを微分した速度,加速度で表される.よって検証用データを生成する際にはこれらの値を決める必要がある.本研究では位置は$[0, \ell]$,速度はハイパーパラメータ$v_{\text{range}}$を用いた$[-v_{\text{range}}, v_{\text{range}}]$,半径はハイパーパラメータ$r_{\text{min}}, r_{\text{max}}$を用いて$[r_{\text{min}}, r_{\text{max}}]$から一様サンプリングして初期化する.また加速度は$0$で初期化する.
    \item 次に,連続的な物体の挙動を再現するために,$dt=0.01$など小さな値を用いて位置,速度,加速度を更新する.このときの更新式は式\ref{eq:sort_transition}と同様に,
    \begin{equation}
        \label{eq:transition}
        \begin{aligned}
            x_{t+1} &= x_t + \dot{x}_t dt + \frac{1}{2} a_t dt^2
            \\v_{t+1} &= \dot{x}_t + a_t dt
            \\a_t &\sim \mathcal{N}(0, A_x)
        \end{aligned}
    \end{equation}
    で計算される.すなわち物体に加えられる外力は正規的なランダム力だとする.ここで冗長性のため$x$に対しての式のみ記している.また加速度に加えられるノイズの分散は$x, y, z$ですべて共通とし,ハイパーパラメータ$\sigma_{\text{noise}}$として設定する.
    \item 共焦点顕微鏡が一枚の二次元画像を取得するのにかかる時間が$t_c$である場合,連続的な時刻が$t_c$の整数倍になるタイミングで画像の取得を行う.この$t / t_c$は\ref{sec:setting}で述べた離散時刻$\tau$に相当する.この画像は\ref{sec:setting}で説明した式\ref{eq:appearant_radius}や式\ref{eq:circle_region}に応じて生成する.またこのときのスライス$s$の値は,離散時刻$\tau$をさらに$n_s$で割った余りである.これにより,$n_s$枚取得するまでは取得のたびにスライスの高さが式\ref{eq:delta_time_and_z}における$\delta z$ずつ高くなり,$n_s$枚に達したところで$s = 0$に戻る.
    \item 手順3の画像取得と同時に,評価に必要な正解ラベルも生成する.まずReIDと状態推定の評価のために検出バウンディングボックスの正解ラベルを生成する.このバウンディングボックスの$(x, y)$には,各物体の連続的な位置$x(t), y(t)$を採用し,$(w, h)$には式\ref{eq:appearant_radius}で示した見かけ上の半径$r_{\text{app}}(t)$の二倍の値を用いる.また状態推定の評価のために,各物体について一つの$\tau$につき一つの状態変数を生成する.状態推定で推定する状態変数は$s = 0$の画像が取得された時刻での状態を示すため,$s = 0$の画像が取得されるタイミングで各物体の位置を出力する.
    \item 検証用データの生成ではランダムな加速度を与えているため,物体が対象領域$\Omega$を出てしまうことも考えられる.すると速度範囲を変えた際などに平等な評価を行うことが困難になるため,領域を出てしまった時点でその分の新しい物体の生成を行う.この生成方法は手順1と同様である.物体のすべての領域が対象領域$\Omega$と共通部分を持たなくなったら領域を出たと判断する.
\end{enumerate}

以降の実験では特記しない限り,$n_b = 10$,$v_{\text{range}} = 0.1$,$r_{\text{min}} = 5$ ,$r_{\text{max}} = 10$,$\sigma_{\text{noise}} = 0.01$,$t_c = 1$,$n_s = 101$を基準の検証用データとして使用する.

\section{提案手法のパイプライン評価}
\label{sec:pipeline_evaluation}

本章では,検証用データを使用して理想的な物体検出をしたときのReIDと状態推定からなるパイプラインの精度評価を行う.またここでは状態推定そのものを評価せず,時刻$t-1$までの物体検出の出力が得られたときの,時刻$t$の状態予測の精度を評価する.実際にはこの状態予測にしたがってマイクロロボットによる細胞の取得が行われるためである.

\subsubsection{実装詳細}
パイプライン評価では表\ref{tab:reidentification_methods}に示す四つの手法を評価する.時間方向の追跡を打ち切る閾値は$\sigma_{\text{sort-act}} = 3$,空間方向の追跡を打ち切る閾値は$\sigma_{\text{d-sort-act}} = 1$とした.またすべてのカルマンフィルタにおいて,システムノイズ分散は$100$,
観測ノイズの大きさは$1.0$,初期時刻における推定状態分散は$\Sigma_0 = 100 * I_n$とする.ここでの$n$は状態変数の長さであり$I_n$は$n$次の単位行列である.また動作環境は,OSがUnuntu 20.04.6 LTS(Focal Fossa)であり,CPUがIntel(R) Xeon(R) Gold 6338 CPU @ 2.00GHzの32コアである.

\subsubsection{比較手法}
本実験ではReIDに\ref{sec:existing_method}で説明したSORTと外観情報を用いずに高精度なReIDを達成するByteTrack\cite{zhang2022bytetrack}を適用したものを比較手法として評価する.これらの比較手法は単一のスライスにしか適用できないため,各スライスで追跡される情報を独立したものとしてReIDの結果を得る.またこれらの比較手法では,単純な状態推定を行う.すなわち,単一のスライス$s$におけるバウンディングボックスの軌跡から予測された時刻$t + 1$のバウンディングボックスが,$[x, y, w, h]$で表されるとき,時刻$t+1$の状態推定値$\bm{s}_{t+1}^{\text{pred}}$を,
\begin{equation}
    \label{eq:naive_state_estimation}
    \bm{s}^{\text{pred}}_{t+1} = \left[x,~ 0,~ y,~ 0,~ s \delta z,~ 0,~ \frac{\max(w, h)}{2}\right]
\end{equation}
として得る.すなわち物体の中心がそのスライス上かつバウンディングボックスの中心に一致し,そのバウンディングボックスが球体の半径を用いて表せられる円に外接しているとして推定している.

\subsubsection{評価指標}

ReIDの状態推定を組み合わせたパイプラインの出力は,時刻$t$において追跡している各物体$i \in \mathcal{I}_t$の時刻$t+1$の状態推定$\bm{s}_{t+1}$である.この状態推定値の推定精度を評価するために,三次元空間における予測球体と正解ラベル球体の共通領域度合いを示すIoUの平均値(以後meanIoUと呼ぶ)を用いる.ある二つの球体$\bm{s}_1 = [x_1, y_1, z_1, r_1]$と$\bm{s}_2 = [x_2, y_2, z_2, r_2]$が存在するとき,IoU$(\bm{s}_1, \bm{s}_2)$は,
\begin{equation}
    \label{eq:sphere_iou}
    \begin{aligned}
        d &= \sqrt{(x_1 - x_2)^2 + (y_1 - y_2)^2 + (z_1 - z_2)^2}
        \\V_{\text{inter}} &= \left\{
            \begin{aligned}
                &\frac{\pi}{12 d} \left\{d - (r_1 + r_2)\right\}^2 \left\{d^2 + 2(r_1 + r_2)d - 3 (r_1 - r_2)^2\right\},& \quad &\textbf{if } d < r_1 + r_2
                \\ & 0,& &\textbf{else} 
            \end{aligned}
        \right.
        \\\text{IoU}(\bm{s}_1, \bm{s}_2) &= \frac{V_{\text{inter}}}{\frac{4}{3}\pi r_1^2 + \frac{4}{3}\pi r_2^2 - V_{\text{inter}}}
    \end{aligned}
\end{equation}
で表される.

meanIoUは,予測された球体ごとに正解ラベル球体とIoUを計算し,これらの平均値を取る.また評価の際には,予測されたある球体が正解ラベルの球体におけるどの球体に対応する予測なのかを知ることができない.よって評価の際にはスコアであるmeanIoUが最大になるようにこの対応付けを行う.すなわち$f_{\text{sim}}$をIoUとしたアルゴリズム\ref{alg:hungarian_matching}を利用する.

\subsubsection{結果}

\begin{figure}[t]
    \begin{subfigure}[t]{0.5\linewidth}
        \centering
        \includegraphics[width=\linewidth]{fig/meanIoUs.pdf}
    \end{subfigure}
    \hfill
    \begin{subfigure}[t]{.5\linewidth}
        \centering
        \includegraphics[width=\linewidth]{fig/part4/singleIoU.pdf}
    \end{subfigure}
    \\    
    \begin{subfigure}[t]{0.5\linewidth}
        \centering
        \includegraphics[width=\linewidth]{fig/prediction_count.pdf} 
    \end{subfigure}
    \hfill
    \begin{subfigure}[t]{.5\linewidth}
        \centering
        \includegraphics[width=\linewidth]{fig/part4/prediction_count_proposed.pdf}
    \end{subfigure}
    \vspace{-3zh}
    \caption[パイプラインにおける既存手法との比較]{パイプラインにおける既存手法との比較:}
    \label{fig:pipeline_assessment}
\end{figure}

図\ref{fig:pipeline_assessment}左上では,すべての時刻において四つすべての提案手法が既存手法の精度を上回っていることが分かる.meanIoUは予測された物体で平均を取っているため,一物体あたりの精度でも大きく打ち勝っているといえる.また予測されている物体数に着目すると(図左下),提案手法に対して既存手法の予測物体数が一桁大きいことが分かる.これはスライスごとに予測を行っているためだと考えられる.これは実際に予測を参考にする場合は,どれを選択すればよいか判断しづらい,という問題に発展する.一方で提案手法では予測物体数を抑えられており,精度と物体数両方に利点があるといえる.

また図右上では,提案手法三つに対してある追跡物体一つのIoUをプロットしている.この図から分かる通り,一物体に対する提案手法それぞれの状態推定精度には違いがない.一方で図右下に着目すると,予測している物体数には差が出ている.また後ほど詳しく議論するが,右上の図のように状態推定精度はある時刻までは一定でそのあと落ち始める.以上の結果から,左上の図で提案手法ごとに精度に差が出る要因の一つとして,予測物体数が小さくなることである物体の精度低下の効果が顕著に出ていることが挙げられる.例えば時刻$60$から時刻$80$あたりでは,SORT + DepthSORTの予測物体数が最も少ない.これにより右上の図のような精度低下分が全体の精度に対して大きくなり,左上のように他の提案手法よりも精度が悪くなったと考えられる.

\section{提案手法の段階的評価}
\label{sec:stage_evaluation}

本章では提案部分であるReID手法と状態推定手法を段階的に分けて精度評価を行う.ReID評価では,一般的なMOTの評価指標と各物体が検出された時刻およびスライスを可視化する方法を用いて,提案手法がスライスを跨いだReIDを達成しているかを確認する.また後続の状態推定評価では,カルマンフィルタにおける事後分布の期待値および分散を確認することで,SliceKalmanFilterが高精度に三次元的な状態を推定できていることを確認する.

    \subsection{ReID手法の評価}
    \label{subsec:reidentification_evaluation}

    提案するReID手法の目的は,三次元的な観測変数を構築するために,スライスを跨いだIDの対応付けを行うことであった.よってこの節では,MOTによく使われる評価指標のみでなく,Detection-Mapという可視化を用いて所望の結果が得られていることを確認する.

        \subsubsection{MOTの評価指標}

        MOTにおける潜在的なエラーは,検出の可否,IDの振り違い,そして検出位置のズレの三つから構成される.本節では理想的な物体検出を想定しているため,この中で本節に用いることが有効であるのはIDの振り違いを評価するものである.よって本節では,ID-switch(以後IDSWと呼ぶ)とAssociation-Accuracy(以後AssAと呼ぶ)という評価指標を用いる.

        MOTの評価指標の説明を行うために,いくつか用語の導入を行う(ただしこれらの用語はMOTの評価指標の一つであるHOTAに関する論文\cite{luiten2021hota}に揃えている).まず検出されたバウンディングボックスをprDets,バウンディングボックスの正解ラベルをgtDetsという集合として表す.またこれらには,どの個体由来であるかを示すprIDsとgtIDsの集合に含まれるIDが割り振られている.すなわちある検出バウンディングボックスprDetには,prID(prDet)が割り振られている.また評価指標を算出する際には,どのprIDがどのgtIDに対応するのか(IDの対応付け)やどのprDetがgtDetに対応するのか(Detの対応付け)などが必要であり,これらは基本的に最終的なスコアを最大化するように対応付けられる.
    
        IDSW\cite{luiten2021hota}は,あるgtIDを持つgtDetsに着目したとき,そのgtDetsに対応付けられたprDetsの中にいくつのprIDが混在しているか(実際にはswitch回数を計算するので個数$-1$)を数え上げたものである.もし理想的なReIDが行われた場合,この値は$0$になる.
    
        またAssA\cite{luiten2021hota}は,前述の三つのエラーをすべて考慮したHOTAを構成する評価指標のうち,当初の目的であるIDの振り違いを評価できる評価指標である.このAssAは,ある対応付けが行われたprDetとgtDetがあるときに,それぞれが持つIDが振られているprDetsとgtDetsに着目し,どれだけその集合が一致しているかを一般的なPrecisionやRecallの考えに則って算出したものを平均したものである.もし理想的なReIDが行われた場合,この値は$1.0$になる.

        \subsubsection{Detection-Mapによる可視化}

        また本節では詳細な議論を行うために,Detection-Mapという図示によってReIDの結果の可視化を行う.このDetection-Mapでは,追跡を行った各IDの物体に対して,そのIDを振られたバウンディングボックスがどの(時刻$t$とスライス$s$で特徴づけられる)画像で検出されたものであるのかを示す.すなわち,各物体がどの画像で検出されたのか,どれだけその物体を追跡できているのかを示すことができる.

        \subsubsection{結果}

        \begin{table}[t]
            \centering
            \caption[ReIDにおける提案手法の比較]{ReIDにおける提案手法の比較:}
            \label{tab:metrics_reidentification}
            \begin{tabular}{l|ccc}
                Methods & Number of Predictions & IDSW & AssA (\%)
                \\\hline \hline
                SORT + DepthSORT & $38.6 ~ (\pm 5.5)$ & $51.8 ~ (\pm 19.9)$ & $92.3 ~ (\pm 3.7)$
                \\ SKF + DepthSORT & $42.8 ~ (\pm 4.6)$ & $61.8 ~ (\pm 17.3)$ & $91.5 ~ (\pm 3.0)$
                \\ only SKF & $44.2 ~ (\pm 4.9)$ & $92.8 ~ (\pm 21.0)$ & $93.6 ~ (\pm 1.8)$
            \end{tabular}
        \end{table}

        \begin{figure}[t]
            \centering
            \includegraphics[width=\linewidth]{fig/detection_maps.pdf}
            \caption[提案するReID手法によるDetection-Mapの一部]{提案するReID手法によるDetection-Mapの一部}
            \label{fig:detection_map}
        \end{figure}

        表\ref{tab:metrics_reidentification}に,三つの提案手法におけるMOTの評価指標,および前節の結果に関連する予測物体数を示す.またReIDにDepthSORTを用いる場合,ほとんどのDetection-Mapは変わらないため,図\ref{fig:detection_map}にはDepthSORTを用いた場合とそうでない場合のDetection-Mapの一部を載せている.

        まず表のAssAに着目すると,どの手法でも$9$割以上の精度を達しており,シードによる分散を考慮するとこれらに大きな違いは見られなかった.また図のDetection-Mapを確認すると,いずれの手法でも大域的に安定したスライドを跨ぐReIDを達成できていることが分かる.ただし,DepthSORTを用いた場合の方がより安定しており,用いない場合は図の下側に示すように少なからず欠損が確認された.

        一方IDSWや予測物体数を比較すると,SORT+DepthSORTが最も安定し,SliceKalmanFilterのみを用いる場合は少し不安定になっていることが分かる.図の下側と合わせて考えると,DepthSORTを用いないことである同一個体由来のバウンディングボックスが分裂してしまい,別個体として追跡され,表に現れているようにIDSWや予測物体数の増加に起因したと考えられる.


    \subsection{状態推定手法の評価}
    \label{subsec:estimation_evaluation}

    

    %%%%%%%%%%%%%%%%%%%%%
    % Naive 比較は後回し!!!! (ここでperformanceも出す) % 
    %%%%%%%%%%%%%%%%%%%%%%%%%%

        \subsubsection{使用する検証用データ}

        ReID手法によって生じる精度差をより詳細に議論するために,物体数$n_b = 1$とした検証用データを使用して評価を行う.また通常の検証用データでは,すべての時刻における物体数を保つために手順5において物体の生成を行っていたが,ここでは生成を行わず,ただ一つの物体に着目した評価を行う.よって検出されるバウンディングボックスはすべて同個体由来のものとして扱い,SliceKalmanFilterに入力する.

        \subsubsection{結果}

        \begin{figure}[t]
            \begin{subfigure}[t]{\linewidth}
                \centering
                \includegraphics[width=\linewidth]{fig/part4/performance_000.pdf}
            \end{subfigure}
            \\
            \begin{subfigure}[t]{\linewidth}
                \centering
                \includegraphics[width=0.5\linewidth]{fig/part4/IoU_seed1234_GT.pdf}
            \end{subfigure}
            \caption[SliceKalmanFilterの精度評価]{SliceKalmanFilterの精度評価:}
            \label{fig:SKF_evaluation}
        \end{figure}

        図\ref{fig:SKF_evaluation}に事後分布における各状態変数の$95\%$信頼区間および正解ラベルの値を示す.また議論のために状態推定の精度を示すmeanIoU(ただし$n_b = 1$なので単一のIoUでもある)も同時に示す.各状態変数の予測分布に着目すると,ほとんど正解ラベルと差がなく,高精度に推定できていることが分かる.ただし$y$のグラフの時刻$60$付近に着目すると,その正解ラベルおよび推定値が対象領域の端である$\ell = 100$付近に近づいており,その時刻以降精度が落ちていることが分かる.また同時に他の状態変数の精度も低下している.

        これは物体が対象領域の端に近づくにつれて検出されるバウンディングボックスが欠けていることが原因として考えられる.このような対象領域の端における精度低下は他の物体でも見られ,反対に端以外における精度低下は見られなかった.すなわちSliceKalmanFilterが画像から見切れる状況を扱いきれておらず,状態推定精度の図のような挙動に起因したと考えられる.

\section{スライス数の変化に対する評価}
\label{sec:number_of_slice_effect}

\ref{sec:setting}では,対象空間を何枚のスライスで捉えるのかを$n_s$で表し,この$n_s$の値によってフレーム間隔とスライス間隔のトレードオフが生じることを説明した.よって本節では,このスライス枚数$n_s$を変化させたときに追跡精度meanIoUにどのような影響を及ぼすのかを検証する.評価方法は,前節で用いたものを引き続き利用する.また提案手法のうちonly SORTは高精度が望まれないことが自明であるため,本節では実験の対象から除外する.

    \subsubsection{データセット}

    \begin{table}[t]
        \centering
        \caption[スライス数変化の評価実験におけるデータセット]{スライス数変化の評価実験におけるデータセット:$t_c = 1.0$として計算している.}
        \label{tab:ns_exp_detail}
        \begin{tabular}{c|ccc}
            スライス数$n_s$ & 離散的な時間長さ & フレーム間隔$dt$ & スライス間隔$\delta z$ 
            \\ \hline \hline
            $11$ & $927$ & $11.0$ & $10.0$
            \\ $21$ & $485$ & $21.0$ & $5.0$
            \\ $51$ & $200$ & $51.0$ & $2.0$
            \\ $101$ & $101$ & $101.0$ & $1.0$
            \\ $201$ & $50$ & $201.0$ & $0.5$
        \end{tabular}
    \end{table}

    二次元画像一枚の取得にかかる連続時間は一定のため,対象とする離散時間長さを一定にすると対象とする連続時間長さが変わってしまう.そこで本実験では,取得する画像数を$101 \times 101 = 10201$に固定し,これをそれぞれの$n_s$で割った余りを離散的な時間長さとしてデモデータを作成する.この詳細を表\ref{tab:ns_exp_detail}に示す.

    \subsubsection{結果}

    \begin{figure}[t]
        \begin{subfigure}[t]{\linewidth}
            \centering
            \includegraphics[width=\linewidth]{fig/part4/meanIoUs_with_ns.pdf}
        \end{subfigure}
        \\
        \begin{subfigure}[t]{.5\linewidth}
            \centering
            \includegraphics[width=\linewidth]{fig/part4/AssA_with_ns.pdf}
        \end{subfigure}
        \begin{subfigure}[t]{.5\linewidth}
            \centering
            \includegraphics[width=\linewidth]{fig/part4/IDSW_with_ns.pdf}
        \end{subfigure}
        \caption[スライス数変化の影響]{スライス数変化の影響:}
        \label{fig:ns_effect}
    \end{figure}

    図\ref{fig:ns_effect}の上側に,各提案手法に対するスライス数の影響,下側にReIDを評価するためのAssAとIDSWに対する影響を示す.すべての提案手法に共通して,$n_s = 11$を除くとスライス数の上昇に応じて状態推定精度が落ちる傾向が見られた.ReIDに対する影響に着目すると,スライス数の上昇に応じてReIDの精度が上がっているため,フレーム間隔$dt$がSliceKalmanFilterによる状態推定の精度を下げてしまっていることが分かる.

    これは逆に考えると,必ずしもすべてのバウンディングボックスを対応付ける必要がなく,フレーム間隔を下げることが状態推定にとって重要であると考えることができる.すなわちSliceKalmanFilterを状態推定に用いる場合は,スライス数をなるべく小さくしてフレーム間隔を縮めたほうがよい,という指針を得ることができる.

    ただし,$n_s = 11$の結果には注意が必要である.状態推定およびReIDどちらの結果でも,$n_s = 11$までスライス数を小さくすると著しく精度が低下している.これは,スライス間隔が物体の大きさのオーダーに近く,個体ごとに検出されるバウンディングボックスが極端に少なくなったことが原因として考えられる.実際に,$n_s = 11$の場合のスライス間隔は$\delta z = 10.0$であり(表\ref{tab:ns_exp_detail}参照),物体の直径は$10 \sim 20$(\ref{sec:demodata}節参照)である.以上の議論より,スライス数は少なくとも,物体の大きさがスライス間隔の$2$倍程度になるように設定することが望まれる.

\section{物体検出を含んだパイプライン評価}
\label{sec:pipleline_evaluation_with_detection}

ここまでの節では,理想的な物体検出を行ったとして検証用データを用いてReIDおよび状態推定を評価してきた.しかしながら実際の物体検出では誤検出や見逃しがあり,これらは後続のReIDや状態推定に影響を及ぼすはずである.よってこの節ではまずはじめに,実際に共焦点顕微鏡から得られるプロトプラスト化した細胞に対する物体検出を評価する(\ref{subsec:detection_of_protoplast}項).そしてそのあと,誤検出や見逃しが追跡に与える影響およびそれらへの対処を述べる(\ref{subsec:FPeffect}項および\ref{subsec:FNeffect}項).

    \subsection{プロトプラスト化した細胞に対する物体検出評価}
    \label{subsec:detection_of_protoplast}

    \subsubsection{実装詳細}

    本実験では,物体検出モデルとしてYOLOv10の六つのサイズ(N/S/M/B/L/X)で評価を行う(原論文\cite{wang2024yolov10}のTable1参照).また微調整(Fine-Tuning)には,共焦点顕微鏡から得られた$865$枚の画像にアノテーションを実施し,そのうち$692$枚を学習データ,残りの$173$枚を検証データとした.微調整のハイパーパラメータについてはYOLOv10の原論文\cite{wang2024yolov10}のTable14と同様である.またクラス数は,遊離した細胞の一つのみに設定した.

    \subsubsection{評価指標}

    本実験では,評価指標として誤検出を示すFalse Positive(FP)と見逃しを示すFalse Negative(FN),そしてそれらと正解を示すTrue Positive(TP)から算出されるF1scoreを用いる.しかしながら物体検出におけるこれらは少し複雑性を伴うため,以下に説明を述べる.

    一般的な混同行列の成分(TP,FP,FN,TN)は,各インスタンス(例えば画像分類であれば各画像を指す)に対して,予測ラベルが正解ラベルと一致しているかどうかで計上される.しかしながら物体検出では,どの予測バウンディングボックスが正解ラベルのバウンディングボックスと対応づいているかが非自明なため,一貫性のある対応付けが必要である.

    そこで物体検出における混同行列の算出では,各正解ラベルのバウンディングボックスそれぞれについて,IoUがある閾値以上であり,その中で正解ラベルのクラスに属する予測クラス確率が最も高いものがその予測として対応付けられる.ただしある予測バウンディングボックスが二つ以上の正解ラベルのバウンディングボックスに対して最大値のクラス確率をもつ場合,IoUが高いほうが優先され,もう片方には次点の予測バウンディングボックスが対応付けられる.またこのIoUの閾値は通常$0.5$に設定される.

    この対応付けが行われた後,対応付けがあった予測バウンディングボックスをTP,対応付けがなかった予測バウンディングボックスをFP,また対応付けがなかった正解ラベルのバウンディングボックスをFNとする.また物体検出において物体でない領域は無限個定義できてしまうため,TNは算出されない.

    これらの混同行列の成分を用いてF1scoreは,
    \begin{equation}
        \label{eq:f1_score}
        \begin{aligned}
            &F1 = \frac{2 p\cdot r}{p + r},
            \\ &\textbf{where} \quad p = \frac{TP}{TP + FP}, ~ r = \frac{TP}{TP + FN}
        \end{aligned}
    \end{equation}
    で表される.ここで$p$は精度(precision),$r$は再現率(recall)と呼ばれ,精度は検出したもののうちどれだけ正解が含まれていたか,再現率は正解であるものをどれだけ検出することができたかを示す評価指標でもある.またこれらの値は,\ref{subsubsec:yolo_components}で説明したように信頼度の閾値によって変化することに注意されたい.

    \subsubsection{結果}

    \begin{figure}[t]
        \centering
        \includegraphics[width=\linewidth]{fig/yolo_f1score.pdf}
        \caption[細胞に対する物体検出評価]{細胞に対する物体検出評価}
        \label{fig:yolo_f1score}
    \end{figure}

    図\ref{fig:yolo_f1score}に結果を示す.モデルサイズの違いに着目すると,最もサイズの大きいXではそれより小さいモデルよりもF1scoreの最大値が小さくなっていることが分かる.これはおそらくモデルサイズに比べてデータセットサイズが小さく,過学習を起こしていると考えられる.

    これらの図より,f1scoreの最大値は$0.8$付近であり,信頼度の閾値は$0.2$付近が良いと判断される.すなわち,少なくともFPおよびFNは二割以上発生すると想定して複数物体追跡を行うことが望まれる.

    \subsection{物体検出の誤検出が及ぼす影響}
    \label{subsec:FPeffect}

    まず問題を切り分け,誤検出(FP)が追跡に及ぼす影響を評価する.本項では,検証用データ作成手順4において,検出に誤検出が含まれる割合を$r_{\text{FP}} \in [0, 1]$として定め,各画像ごとに誤検出割合が$r_{\text{FP}}$になるように偽のバウンディングボックスを生成する.このときバウンディングボックスの位置$(x,y)$は$[0, \ell]$から一様サンプリングし,サイズ$(w, h)$は$[r_{\text{min}}, r_{\text{max}}]$から一様サンプリングしたのち二倍した値を共通して与える.またこのバウンディングボックスが領域$[0, 1]^2$を超える場合は,$[0, 1]$にクランプする.

        \subsubsection{結果}

        \begin{figure}[t]
            \begin{subfigure}[t]{\linewidth}
                \centering
                \includegraphics[width=\linewidth]{fig/part4/meanIoUs_with_fp.pdf}
            \end{subfigure}
            \\
            \begin{subfigure}[t]{\linewidth}
                \centering
                \includegraphics[width=\linewidth]{fig/part4/count_with_fp.pdf}
            \end{subfigure}
            \\
            \begin{subfigure}[t]{\linewidth}
                \centering
                \includegraphics[width=\linewidth]{fig/part4/reid_with_fp.pdf}
            \end{subfigure}
            \caption[物体検出のFPが追跡に及ぼす影響]{物体検出のFPが追跡に及ぼす影響:}
            \label{fig:FP_effect}
        \end{figure}

        誤検出率$r_{\text{FP}}$を$0.1$から$0.5$まで変化させたときの状態変数およびReIDに及ぼす影響を図\ref{fig:FP_effect}に示す.まず状態推定ついては,FPの割合を変更させたときに推定精度の挙動は変わるものの,全体的な変化は見られなかった.これはすなわち,物体検出のFPとFNのトレードオフを考える際に,FNを優先して信頼度スコアの閾値を低くすれば良いことを示唆している.

        一方ReIDの精度に着目すると,AssAはあまり変化しないもののIDSWでは変化が見られた.これはつまり,ある個体由来のバウンディングボックスが複数の物体として分裂して認識されているが,ほとんどのバウンディングボックスが一つのIDに割り当てられ,分裂したバウンディングボックスが少量であったと推測できる.すなわちReIDにおいてもFPの大きな影響がないことが分かる.

        これらの結果は,ほとんどの対応付けをIoUベースで実施していることが理由として考えられる.IoUをベースにする場合,たとえFPの割合が大きいとしても,正しい対応付けに影響するのは,TPのバウンディングボックスに対してIoUが$0$より大きいFPのバウンディングボックスのみである.よってFPの割合が大きくてもあまり精度に影響が出なかったと考えれる.

    \subsection{物体検出の見逃しが及ぼす影響}
    \label{subsec:FNeffect}

    次に見逃し(FN)が追跡に及ぼす影響を評価する.本項では,検証用データ作成手順4において,検出を見逃す確率を$r_{\text{fn}} \in [0, 1]$として定め,その確率で各バウンディングボックスの正解ラベルを削除する.そしてこの見逃しを含んだバウンディングボックス群を物体検出後の入力としてReIDおよび状態推定を評価する.

        \subsubsection{結果1 物体検出FNが追跡に及ぼす影響}

        \begin{figure}[t]
            \begin{subfigure}[t]{\linewidth}
                \centering
                \includegraphics[width=\linewidth]{fig/part4/meanIoUs_with_fn.pdf}
            \end{subfigure}
            \\
            \begin{subfigure}[t]{\linewidth}
                \centering
                \includegraphics[width=\linewidth]{fig/part4/count_with_fn.pdf}
            \end{subfigure}
            \caption[物体検出のFNが追跡に及ぼす影響]{物体検出のFNが追跡に及ぼす影響:}
            \label{fig:FN_effect}
        \end{figure}

        図\ref{fig:FN_effect}に物体検出のFNが状態推定(上)および予測物体数(下)に及ぼす影響を示す.まずReIDにDetphSORTを使用しない場合(only SKF)では,FNの割合にあまり影響を受けず,状態推定の精度および予測数は安定している.一方DephtSORTを用いる二手法は,予測物体数においてどちらとも影響を受けており,特にSKF+DepthSORTでは顕著な影響が見られる.これはDepthSORTを行う際に,追跡を止める打ち切り閾値$\sigma_{\text{act}}$を$1$に設定したことが起因し,追跡が途切れたことが理由として考えられる.

        一方状態推定の精度に影響を及ぼしていないのは,SliceKalmanFilterが欠けた観測変数からでもある程度の精度をもって推定できているためだと推測できる.この挙動は,\ref{subsec:number_of_slice_effect}節において,スライス数が少なくても状態推定の精度を保っていた挙動に似ている.もしSliceKalmanFilterがそのような,欠けた観測変数からも精度を保った推定が可能であるならば,SORT+DepthSORTの状態推定が不安定であるのは,別個体由来のバウンディングボックスが対応付けられたためと類推できる.すなわち,二次元情報をベースにした予測によって別個体由来のバウンディングボックスが対応付けられてしまい,SliceKalmanFilterの精度を不安定にさせていると考えられる.

        \subsubsection{結果2 スライス方向の打ち切り閾値変更による対処}

        \begin{figure}[t]
            \begin{subfigure}[t]{\linewidth}
                \centering
                \includegraphics[width=\linewidth]{fig/part4/meanIoUs_with_thres.pdf}
            \end{subfigure}
            \\
            \begin{subfigure}[t]{\linewidth}
                \centering
                \includegraphics[width=\linewidth]{fig/part4/count_with_thres.pdf}
            \end{subfigure}
            \\
            \begin{subfigure}[t]{\linewidth}
                \centering
                \includegraphics[width=\linewidth]{fig/part4/reid_with_thres.pdf}
            \end{subfigure}
            \caption[DepthSORTにおける打ち切り閾値による対処]{DepthSORTにおける打ち切り閾値による対処:}
            \label{fig:handling_with_thres}
        \end{figure}

        結果1では,DepthSORTにおける打ち切り閾値が$\sigma_{\text{act}} = 1$であるために,DepthSORTを用いた追跡に影響を及ぼしていると述べた.そこで結果2では,FNの割合を$r_{\text{fn}} = 0.5$に固定し,この閾値を変化させた場合の精度を評価する.この結果を図\ref{fig:handling_with_thres}に示す.

        図中段の予測物体数のグラフに着目すると,打ち切り閾値を大きくすることでかなり物体数を抑えられていることが分かる.また閾値を大きくした場合には,状態推定の精度が少なからず安定している.よって,検出の見逃し(FN)が生じる場合は,ある程度DepthSORTの打ち切り閾値を大きくするとよい.

        ただし,打ち切り閾値を大きくしすぎると反対に精度の低下を招くことに注意されたい.図の下側では,打ち切り閾値を大きくしすぎた場合のReIDの精度を載せている.図に示すとおり,AssA,IDSWともに$\sigma_{\text{act}} = 5$付近以降,精度が低下している.これは打ち切り閾値が大きくなることで,別個体由来の対応付けが起きてしまったことが原因として考えられる.

        以上の議論より,物体検出の見逃し(FN)が生じる場合,DepthSORTを使用するのであればその打ち切り閾値の調整が必要である.一方で,DepthSORTを使用しないonly SKFで追跡を行うのであれば,ほとんどFNの影響を受けず,考慮する必要がないことが示された.

