\thispagestyle{fancy2}

本論文では,共焦点顕微鏡から得られる複数のスライスにおける二次元画像の時系列データを入力とした複数物体追跡を行うために,DepthSORTというReID手法とSliceKalmanFilterという新たなカルマンフィルタの設計を行った.この二つの提案手法により,既存の複数物体追跡の手法を複数の二次元画像入力にも対応可能にし,検出された情報からは得られない状態変数の推定を可能にした.

われわれは検証用データの計算機実験において,複数のスライスにおける検出情報を統合することでより高精度な三次元的な状態推定を可能にすること,そしてその状態推定を精度良く行うためのスライス間隔と時間間隔のトレードオフの最適なポイントを見つけることができることを示した.

またこの提案手法を組み込んだ位置情報付き一細胞分取システムを用いて,実際にプロトプラスト化した細胞の自動分取に成功した.われわれはこのシステムにより,遺伝子発現解析に重要な空間的分解能を保持したままのscRNA-seqに取り組むことができ,植物の再生力の解明に大きく貢献できると信じている.

\chapter*{謝辞}

本研究を進めるにあたり,数多くの方々にご支援とご協力を頂きました.ここに感謝の意を表すとともに,深くお礼申し上げます.

名古屋大学大学院工学研究科機械システム工学専攻 竹内一郎教授には,本研究の機会を与えていただき,終始的確なご指導を頂きました.また名古屋大学大学院工学研究科機械システム工学専攻 田地宏一准教授には,何度も研究のご支援・ご意見を頂きました.深く感謝申し上げます.

また理化学研究所 花田博幸特別研究員には,日頃の研究活動全般において事細かにご指導を頂き,大変お世話になりました.深く感謝申し上げます.

また名古屋工業大学 烏山・稲津研究室および名古屋大学 竹内・田地研究室の皆様方には,日頃から様々な議論を交わし,本研究にも少ながらずご協力を頂きました.また研究室の同僚として,素晴らしい研究活動の場を支えていただきました.厚くお礼申し上げます.

最後に,日頃から常に支えてくださった家族と友人たちに感謝します.
