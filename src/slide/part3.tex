\section{提案手法}

\begin{frame}[noframenumbering]{目次}
    \tableofcontents[currentsection]
\end{frame}

\begin{frame}{カルマンフィルタによる三次元的な位置予測}
    \begin{itemize}
        \item \uline{カルマンフィルタ} \cite{bishop2001introduction}
        \begin{itemize}
            \item センサー情報(\skyblue{観測変数})からシステム内部情報(\red{状態変数})を推定
            \item ある時刻までの観測情報から,それ以降の状態を予測することも可能
        \end{itemize}
        \item[\phantom{}] \phantom{新たに,\red{\textbf{Slice Kalman Filter}}を設計}
        \begin{itemize}
            \item[\phantom{}] \phantom{拡張カルマンフィルタ:非線形な関係の扱いを可能に(omitted)}
            \item[\phantom{}] \phantom{可変的な観測変数:毎時刻変化する観測の次元数に対応(omitted)}
        \end{itemize}
    \end{itemize}
    \myfigure{slide/5-1.pdf}
\end{frame}
\begin{frame}[noframenumbering]{カルマンフィルタによる三次元的な位置予測}
    \begin{itemize}
        \item \uline{カルマンフィルタ} \cite{bishop2001introduction}
        \begin{itemize}
            \item センサー情報(\skyblue{観測変数})からシステム内部情報(\red{状態変数})を推定
            \item ある時刻までの観測情報から,それ以降の状態を予測することも可能
        \end{itemize}
        \item 新たに,\red{\textbf{Slice Kalman Filter}}を設計
        \begin{itemize}
            \item 拡張カルマンフィルタ:非線形な関係の扱いを可能に(omitted)
            \item 可変的な観測変数:毎時刻変化する観測の次元数に対応(omitted)
        \end{itemize}
    \end{itemize}
    \myfigure{slide/5-2.pdf}
\end{frame}
\begin{frame}[noframenumbering]{カルマンフィルタによる三次元的な位置予測}
    \begin{itemize}
        \item \uline{カルマンフィルタ} \cite{bishop2001introduction}
        \begin{itemize}
            \item センサー情報(\skyblue{観測変数})からシステム内部情報(\red{状態変数})を推定
            \item ある時刻までの観測情報から,それ以降の状態を予測することも可能
        \end{itemize}
        \item 新たに,\red{\textbf{Slice Kalman Filter}}を設計
        \begin{itemize}
            \item 拡張カルマンフィルタ:非線形な関係の扱いを可能に(omitted)
            \item 可変的な観測変数:毎時刻変化する観測の次元数に対応(omitted)
        \end{itemize}
    \end{itemize}
    \myfigure{slide/5-3.pdf}
\end{frame}
\begin{frame}[noframenumbering]{カルマンフィルタによる三次元的な位置予測}
    \begin{itemize}
        \item \uline{カルマンフィルタ} \cite{bishop2001introduction}
        \begin{itemize}
            \item センサー情報(\skyblue{観測変数})からシステム内部情報(\red{状態変数})を推定
            \item ある時刻までの観測情報から,それ以降の状態を予測することも可能
        \end{itemize}
        \item 新たに,\red{\textbf{Slice Kalman Filter}}を設計
        \begin{itemize}
            \item 拡張カルマンフィルタ:非線形な関係の扱いを可能に(omitted)
            \item 可変的な観測変数:毎時刻変化する観測の次元数に対応(omitted)
        \end{itemize}
    \end{itemize}
    \myfigure{slide/5-4.pdf}
\end{frame}

\begin{frame}{既存技術:Tracking-by-Detection}
    \begin{block}{Tracking-by-Detection \cite{luo2021multiple}}
        \begin{enumerate}
            \item 各画像を入力して,画像に映る物体をバウンディングボックス(\red{$\square$})で検出
            \\\ra (e.g.) YOLO:約$10^2$fps \cite{redmon2016you,wang2024yolov10}
            \item 各バウンディングボックスに,個体を識別するためのIDを割り振り(ReID)
            \\\ra (e.g) SORT:約$3\times10^2$fps \cite{bewley2016simple,wojke2017simple,du2023strongsort}
        \end{enumerate}        
    \end{block}
    \myfigure{slide/6-1.pdf}
\end{frame}
\begin{frame}[noframenumbering]{既存技術:Tracking-by-Detection}
    \begin{block}{Tracking-by-Detection \cite{luo2021multiple}}
        \begin{enumerate}
            \item 各画像を入力して,画像に映る物体をバウンディングボックス(\red{$\square$})で検出
            \\\ra (e.g.) YOLO:約$10^2$fps \cite{redmon2016you,wang2024yolov10}
            \item 各バウンディングボックスに,個体を識別するためのIDを割り振り(ReID)
            \\\ra (e.g) SORT:約$3\times10^2$fps \cite{bewley2016simple,wojke2017simple,du2023strongsort}
        \end{enumerate}        
    \end{block}
    \myfigure{slide/6-2.pdf}
\end{frame}
\begin{frame}[noframenumbering]{既存技術:Tracking-by-Detection}
    \begin{block}{Tracking-by-Detection \cite{luo2021multiple}}
        \begin{enumerate}
            \item 各画像を入力して,画像に映る物体をバウンディングボックス(\red{$\square$})で検出
            \\\ra (e.g.) YOLO:約$10^2$fps \cite{redmon2016you,wang2024yolov10}
            \item 各バウンディングボックスに,個体を識別するためのIDを割り振り(ReID)
            \\\ra (e.g) SORT:約$3\times10^2$fps \cite{bewley2016simple,wojke2017simple,du2023strongsort}
        \end{enumerate}        
    \end{block}
    \myfigure{slide/6-3.pdf}
\end{frame}

\begin{frame}{本問題における既存技術の適用}
    \begin{itemize}
        \item 既存技術を適用するためにはどうしたらよいか?
    \end{itemize}
    \myfigure[0.7]{slide/7-1.pdf}
    \phantom{\ra しかしながらこの場合$\dots$}
    \vs
    \begin{columns}
        \begin{column}{.5\linewidth}
            \phantom{\uline{理想のReID}}
            \myfigure{slide/8-0.pdf}
        \end{column}
        \begin{column}{.5\linewidth}
            \phantom{\uline{実際に得られるReID}}
            \myfigure{slide/9-0.pdf}
        \end{column}
    \end{columns}
\end{frame}
\begin{frame}[noframenumbering]{本問題における既存技術の適用}
    \begin{itemize}
        \item 既存技術を適用するためにはどうしたらよいか?
    \end{itemize}
    \myfigure[0.7]{slide/7-2.pdf}
    \phantom{\ra しかしながらこの場合$\dots$}
    \vs
    \begin{columns}
        \begin{column}{.5\linewidth}
            \phantom{\uline{理想のReID}}
            \myfigure{slide/8-0.pdf}
        \end{column}
        \begin{column}{.5\linewidth}
            \phantom{\uline{実際に得られるReID}}
            \myfigure{slide/9-0.pdf}
        \end{column}
    \end{columns}
\end{frame}
\begin{frame}[noframenumbering]{本問題における既存技術の適用}
    \begin{itemize}
        \item 既存技術を適用するためにはどうしたらよいか?
    \end{itemize}
    \myfigure[.7]{slide/7-2.pdf}
    \ra しかしながらこの場合$\dots$
    \vs
    \begin{columns}
        \begin{column}{.5\linewidth}
            \uline{理想のReID}
            \myfigure{slide/8-1.pdf}
        \end{column}
        \begin{column}{.5\linewidth}
            \uline{実際に得られるReID}
            \myfigure{slide/9-1.pdf}
        \end{column}
    \end{columns}
\end{frame}

\begin{frame}{DepthSORTによるスライドを跨いだReID}
    \begin{itemize}
        \item スライスを跨いだ,ReIDを行いたい
        \begin{itemize}
            \item 既存技術:他のスライスでは,別個体として追跡してしまう
            \item[\phantom{}] \phantom{提案手法:\red{\textbf{Depth-SORT}}によって,\red{高さ方向に追跡}を行う}
        \end{itemize}
    \end{itemize}
    \myfigure{slide/10-0.pdf}
\end{frame}
\begin{frame}[noframenumbering]{DepthSORTによるスライドを跨いだReID}
    \begin{itemize}
        \item スライスを跨いだ,ReIDを行いたい
        \begin{itemize}
            \item 既存技術:他のスライスでは,別個体として追跡してしまう
            \item[\red{$\blacktriangleright$}] 提案手法:\red{\textbf{Depth-SORT}}によって,\red{高さ方向に追跡}を行う
        \end{itemize}
    \end{itemize}
    \myfigure{slide/10-1.pdf}
\end{frame}
\begin{frame}[noframenumbering]{DepthSORTによるスライドを跨いだReID}
    \begin{itemize}
        \item スライスを跨いだ,ReIDを行いたい
        \begin{itemize}
            \item 既存技術:他のスライスでは,別個体として追跡してしまう
            \item[\red{$\blacktriangleright$}] 提案手法:\red{\textbf{Depth-SORT}}によって,\red{高さ方向に追跡}を行う
        \end{itemize}
    \end{itemize}
    \myfigure{slide/10-3.pdf}
\end{frame}
\begin{frame}[noframenumbering]{DepthSORTによるスライドを跨いだReID}
    \begin{itemize}
        \item スライスを跨いだ,ReIDを行いたい
        \begin{itemize}
            \item 既存技術:他のスライスでは,別個体として追跡してしまう
            \item[\red{$\blacktriangleright$}] 提案手法:\red{\textbf{Depth-SORT}}によって,\red{高さ方向に追跡}を行う
        \end{itemize}
    \end{itemize}
    \myfigure{slide/10-4.pdf}
\end{frame}
\begin{frame}[noframenumbering]{DepthSORTによるスライドを跨いだReID}
    \begin{itemize}
        \item スライスを跨いだ,ReIDを行いたい
        \begin{itemize}
            \item 既存技術:他のスライスでは,別個体として追跡してしまう
            \item[\red{$\blacktriangleright$}] 提案手法:\red{\textbf{Depth-SORT}}によって,\red{高さ方向に追跡}を行う
        \end{itemize}
    \end{itemize}
    \myfigure{slide/10-5.pdf}
\end{frame}
