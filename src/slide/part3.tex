\section{提案手法}

% \begin{frame}[noframenumbering]{既存技術:Multi Object Tracking}
%     \begin{itemize}
%         \item 口頭
%         \begin{itemize}
%             \item 複数物体を追跡するフレームワーク,自動車や歩行者など
%             \item 画像を入力したら,検出して,IDを振っていくことで達成する
%         \end{itemize}
%         \item 図:通常のMOT,下側に途切れたり精度が悪くなる図(可能であれば)
%         \\\ra 適用の仕方がマストかも??
%     \end{itemize}
% \end{frame}
\begin{frame}{既存技術:Multi Object Tracking}
    \begin{block}{Tracking-by-Detection}
        \begin{enumerate}
            \item 各時刻の画像において,画像内に映る物体を検出
            \item 検出された各情報に対して,個体を識別するためのIDを割り振り
        \end{enumerate}
    \end{block}
    \myfigure{slide/slide5-1.pdf}
\end{frame}
\begin{frame}[noframenumbering]{既存技術:Multi Object Tracking}
    \begin{block}{Tracking-by-Detection}
        \begin{enumerate}
            \item 各時刻の画像において,画像内に映る物体を検出
            \item 検出された各情報に対して,個体を識別するためのIDを割り振り
        \end{enumerate}
    \end{block}
    \myfigure{slide/slide5-2.pdf}
\end{frame}
\begin{frame}[noframenumbering]{既存技術:Multi Object Tracking}
    \begin{block}{Tracking-by-Detection}
        \begin{enumerate}
            \item 各時刻の画像において,画像内に映る物体を検出
            \item 検出された各情報に対して,個体を識別するためのIDを割り振り
        \end{enumerate}
    \end{block}
    \myfigure{slide/slide5-3.pdf}
\end{frame}

% \begin{frame}[noframenumbering]{既存技術の適用上生じる課題}
%     \begin{itemize}
%         \item 口頭
%         \begin{itemize}
%             \item これを本問題に適用するには,単一の平面をそれぞれ,独立に処理する方法がある
%             \item しかしそれでは,縦に動く細胞が途切れたり,三次元位置の予測精度に影響が出る
%             \item 以下,blockで問題点
%             \item 縦方向に動く細胞の追跡が途切れる
%             \item 三次元推定の精度保証が自明でない
%         \end{itemize}
%         \item 図:適用方法
%     \end{itemize}
% \end{frame}
\begin{frame}{既存技術の適用上生じる課題}
    \begin{itemize}
        \item 既存技術(Tracking-by-Detection,\ TBD)を本研究に適用するためには?
        \\\ra 各高さ(スライス)ごとに,独立した別系列データとして,TBDに入力
    \end{itemize}
    \myfigure[0.8]{slide/slide6-1.pdf}
    \vs
    \phantom{しかしながら,本研究においてはいくつか問題点}
    \begin{enumerate}
        \item[\phantom{}] \phantom{\uline{元住所の特定}}
        \\\phantom{\ra 細胞が高さ方向に運動すると,追跡が途切れる}
        \item[\phantom{}] \phantom{\uline{三次元位置の推定および予測}}
        \\\phantom{\ra スライスを限定してるため,三次元的な情報が欠けている}
    \end{enumerate}
\end{frame}
\begin{frame}{既存技術の適用上生じる課題}
    \begin{itemize}
        \item 既存技術(Tracking-by-Detection,\ TBD)を本研究に適用するためには?
        \\\ra 各高さ(スライス)ごとに,独立した別系列データとして,TBDに入力
    \end{itemize}
    \myfigure[0.8]{slide/slide6-2.pdf}
    \phantom{しかしながら,本研究においてはいくつか問題点}
    \vs
    \begin{enumerate}
        \item[\phantom{}] \phantom{\uline{元住所の特定}}
        \\\phantom{\ra 細胞が高さ方向に運動すると,追跡が途切れる}
        \item[\phantom{}] \phantom{\uline{三次元位置の推定および予測}}
        \\\phantom{\ra スライスを限定してるため,三次元的な情報が欠けている}
    \end{enumerate}
\end{frame}
\begin{frame}{既存技術の適用上生じる課題}
    \begin{itemize}
        \item 既存技術(Tracking-by-Detection,\ TBD)を本研究に適用するためには?
        \\\ra 各高さ(スライス)ごとに,独立した別系列データとして,TBDに入力
    \end{itemize}
    \myfigure[0.8]{slide/slide6-3.pdf}
    \phantom{しかしながら,本研究においてはいくつか問題点}
    \vs
    \begin{enumerate}
        \item[\phantom{}] \phantom{\uline{元住所の特定}}
        \\\phantom{\ra 細胞が高さ方向に運動すると,追跡が途切れる}
        \item[\phantom{}] \phantom{\uline{三次元位置の推定および予測}}
        \\\phantom{\ra スライスを限定してるため,三次元的な情報が欠けている}
    \end{enumerate}
\end{frame}
\begin{frame}{既存技術の適用上生じる課題}
    \begin{itemize}
        \item 既存技術(Tracking-by-Detection,\ TBD)を本研究に適用するためには?
        \\\ra 各高さ(スライス)ごとに,独立した別系列データとして,TBDに入力
    \end{itemize}
    \myfigure[0.8]{slide/slide6-3.pdf}
    \vs
    しかしながら,本研究においてはいくつか問題点
    \begin{enumerate}
        \item \uline{元住所の特定}
        \\\ra 細胞が高さ方向に運動すると,追跡が途切れる
        \item \uline{三次元位置の推定および予測}
        \\\ra スライスを限定してるため,三次元的な情報が欠けている
    \end{enumerate}
\end{frame}

% \begin{frame}[noframenumbering]{提案手法の概要}
%     \begin{itemize}
%         \item 口頭
%         \begin{itemize}
%             \item 三段階で構成
%             \item 物体検出,IDを割り振り,カルマンフィルタで予測する
%             \item DepthSORTとSKFを提案する
%         \end{itemize}
%         \item 図:ReIDを変更予定
%     \end{itemize}
% \end{frame}
\begin{frame}{提案手法の概要}
    \small
    \begin{block}{Overview of Proposed Method}
        \begin{enumerate}
            \item 物体検出:各画像を検出器に入力して物体の位置を捕捉(既存技術と同様)
            \item IDの割り振り:各検出情報に\red{スライスを跨いで}IDを割り振り
            \\\ra \textbf{\red{Depth-SORT}}によって実現
            \item 位置推定・予測:カルマンフィルタによって各細胞の三次元位置を推定・予測
            \\\ra \textbf{\red{Slice Kalman filter}}によって実現
        \end{enumerate}
    \end{block}
    \myfigure[0.7]{slide/slide7.pdf}
\end{frame}

% \begin{frame}[noframenumbering]{提案手法1:スライスを跨いだIDの割り振り}
%     \begin{itemize}
%         \item 口頭:
%         \begin{itemize}
%             \item スライスを跨いだIDを割り振る
%             \item 縦向きの動きに追従することができる,推定のもと
%         \end{itemize}
%         \item 図:跨いだReIDの図(全スライドと対応するように)
%         \item 図:深さ方向に追跡を行うことで,どれが同じ個体由来であるかを知ることができる
%     \end{itemize}
% \end{frame}
\begin{frame}{提案手法1:スライスを跨いだIDの割り振り}
    \begin{block}{Alogrithm of ID Assignment}
        \begin{enumerate}
            \item 検出された各情報を,由来する個体ごとに仕分ける(by \textbf{\red{Depth-SORT}})
            \item 仕分けられた検出情報と,いままでの軌跡を比較 
            \\\ra 同一個体由来のものにIDを付与 or 新しい物体には新しいID
        \end{enumerate}
    \end{block}
    \myfigure{slide/slide8-1.pdf}
\end{frame}
\begin{frame}[noframenumbering]{提案手法1:スライスを跨いだIDの割り振り}
    \begin{block}{Alogrithm of ID Assignment}
        \begin{enumerate}
            \item 検出された各情報を,由来する個体ごとに仕分ける(by \textbf{\red{Depth-SORT}})
            \item 仕分けられた検出情報と,いままでの軌跡を比較 
            \\\ra 同一個体由来のものにIDを付与 or 新しい物体には新しいID
        \end{enumerate}
    \end{block}    
    \myfigure{slide/slide8-2.pdf}
\end{frame}
\begin{frame}[noframenumbering]{提案手法1:スライスを跨いだIDの割り振り}
    \begin{block}{Alogrithm of ID Assignment}
        \begin{enumerate}
            \item 検出された各情報を,由来する個体ごとに仕分ける(by \textbf{\red{Depth-SORT}})
            \item 仕分けられた検出情報と,いままでの軌跡を比較 
            \\\ra 同一個体由来のものにIDを付与 or 新しい物体には新しいID
        \end{enumerate}
    \end{block}    
    \myfigure{slide/slide8-1.pdf}
\end{frame}
\begin{frame}[noframenumbering]{提案手法1:スライスを跨いだIDの割り振り}
    \begin{block}{Alogrithm of ID Assignment}
        \begin{enumerate}
            \item 検出された各情報を,由来する個体ごとに仕分ける(by \textbf{\red{Depth-SORT}})
            \item 仕分けられた検出情報と,いままでの軌跡を比較 
            \\\ra 同一個体由来のものにIDを付与 or 新しい物体には新しいID
        \end{enumerate}
    \end{block}
    \myfigure{slide/slide8-3.pdf}
\end{frame}
\begin{frame}[noframenumbering]{提案手法1:スライスを跨いだIDの割り振り}
    \begin{block}{Alogrithm of ID Assignment}
        \begin{enumerate}
            \item 検出された各情報を,由来する個体ごとに仕分ける(by \textbf{\red{Depth-SORT}})
            \item 仕分けられた検出情報と,いままでの軌跡を比較 
            \\\ra 同一個体由来のものにIDを付与 or 新しい物体には新しいID
        \end{enumerate}
    \end{block}
    \myfigure{slide/slide8-1.pdf}
\end{frame}
\begin{frame}[noframenumbering]{提案手法1:スライスを跨いだIDの割り振り}
    \begin{block}{Alogrithm of ID Assignment}
        \begin{enumerate}
            \item 検出された各情報を,由来する個体ごとに仕分ける(by \textbf{\red{Depth-SORT}})
            \item 仕分けられた検出情報と,いままでの軌跡を比較 
            \\\ra 同一個体由来のものにIDを付与 or 新しい物体には新しいID
        \end{enumerate}
    \end{block}
    \myfigure{slide/slide8-4.pdf}
\end{frame}
\begin{frame}[noframenumbering]{提案手法1:スライスを跨いだIDの割り振り}
    \begin{block}{Alogrithm of ID Assignment}
        \begin{enumerate}
            \item 検出された各情報を,由来する個体ごとに仕分ける(by \textbf{\red{Depth-SORT}})
            \item 仕分けられた検出情報と,いままでの軌跡を比較 
            \\\ra 同一個体由来のものにIDを付与 or 新しい物体には新しいID
        \end{enumerate}
    \end{block}
    \myfigure{slide/slide8-5.pdf}
\end{frame}
\begin{frame}[noframenumbering]{提案手法1:スライスを跨いだIDの割り振り}
    \begin{block}{Alogrithm of ID Assignment}
        \begin{enumerate}
            \item 検出された各情報を,由来する個体ごとに仕分ける(by \textbf{\red{Depth-SORT}})
            \item 仕分けられた検出情報と,いままでの軌跡を比較 
            \\\ra 同一個体由来のものにIDを付与 or 新しい物体には新しいID
        \end{enumerate}
    \end{block}
    \myfigure{slide/slide8-6.pdf}
\end{frame}
\begin{frame}[noframenumbering]{提案手法1:スライスを跨いだIDの割り振り}
    \begin{block}{Alogrithm of ID Assignment}
        \begin{enumerate}
            \item 検出された各情報を,由来する個体ごとに仕分ける(by \textbf{\red{Depth-SORT}})
            \item 仕分けられた検出情報と,いままでの軌跡を比較 
            \\\ra 同一個体由来のものにIDを付与 or 新しい物体には新しいID
        \end{enumerate}
    \end{block}
    \myfigure{slide/slide8-7.pdf}
\end{frame}

% \begin{frame}[noframenumbering]{提案手法2:三次元位置の状態予測}
%     \begin{itemize}
%         \item 口頭:
%         \begin{itemize}
%             \item 次に,スライスを跨いだ複数のバウンディングボックスから推定をする
%             \item カルマンフィルタによって行う
%             \item 貢献として,EKFと可変的な観測変数を用いていることもいう
%             \item or カルマンフィルタの導入から話す
%         \end{itemize}
%         \item 図:行っていることを図示化?
%     \end{itemize}
% \end{frame}
\begin{frame}{提案手法2:三次元位置の状態予測}
    \begin{itemize}
        \item \uline{カルマンフィルタ}
        \begin{itemize}
            \item センサーから得られる情報(\skyblue{観測変数})から,システム内部(\red{状態変数})を推定
            \item ある時刻までの情報から,それ以降の状態も予測可能
        \end{itemize}
    \end{itemize}
    \begin{itemize}
        \item[\phantom{}] \phantom{新たに,\red{\textbf{Slice Kalman Filter}}を設計}
        \begin{itemize}
            \item[\phantom{}] \phantom{拡張カルマンフィルタ:非線形な関係の扱いを可能に}
            \item[\phantom{}] \phantom{可変的な観測変数:毎時刻に変化する観測変数の次元数に対応}
        \end{itemize}
    \end{itemize} 
    \myfigure{slide/slide9-1.pdf}
\end{frame}
\begin{frame}[noframenumbering]{提案手法2:三次元位置の状態予測}
    \begin{itemize}
        \item \uline{カルマンフィルタ}
        \begin{itemize}
            \item センサーから得られる情報(\skyblue{観測変数})から,システム内部(\red{状態変数})を推定
            \item ある時刻までの情報から,それ以降の状態も予測可能
        \end{itemize}
    \end{itemize}   
    \begin{itemize}
        \item 新たに,\red{\textbf{Slice Kalman Filter}}を設計
        \begin{itemize}
            \item 拡張カルマンフィルタ:非線形な関係の扱いを可能に
            \item 可変的な観測変数:毎時刻に変化する観測変数の次元数に対応
        \end{itemize}
    \end{itemize} 
    \myfigure{slide/slide9-2.pdf}
\end{frame}

% \begin{frame}[noframenumbering]{提案手法の全体像}
%     \begin{itemize}
%         \item 口頭
%         \begin{itemize}
%             \item 
%         \end{itemize}
%         \item 図
%     \end{itemize}
% \end{frame}
\begin{frame}{提案手法の全体像}
    \uline{時刻$t$における処理}
    \myfigure[0.8]{slide/slide10.pdf}
\end{frame}
