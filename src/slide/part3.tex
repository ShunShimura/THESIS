\section{提案手法}

\begin{frame}[noframenumbering]{既存技術:Multi Object Tracking}
    \begin{itemize}
        \item 口頭
        \begin{itemize}
            \item 複数物体を追跡するフレームワーク,自動車や歩行者など
            \item 画像を入力したら,検出して,IDを振っていくことで達成する
            \item これを本問題に適用するには,単一の平面をそれぞれ,独立に処理する方法がある
            \item しかしそれでは,縦に動く細胞が途切れたり,三次元位置の予測精度に影響が出る
            \item 提案手法へ
        \end{itemize}
        \item 図:通常のMOT,下側に途切れたり精度が悪くなる図(可能であれば)
        \\\ra 適用の仕方がマストかも??
    \end{itemize}
\end{frame}
\begin{frame}{既存技術:Multi Object Tracking}
    \myfigure{slide/slide5.pdf}
\end{frame}

\begin{frame}[noframenumbering]{既存技術の適用上生じる課題}
    \begin{itemize}
        \item 口頭
        \begin{itemize}
            \item 
        \end{itemize}
        \item 図
    \end{itemize}
\end{frame}


\begin{frame}[noframenumbering]{提案手法1:スライスを跨いだIDの割り振り}
    \begin{itemize}
        \item 口頭:
        \begin{itemize}
            \item スライスを跨いだIDを割り振る
            \item 縦向きの動きに追従することができる,推定のもと
        \end{itemize}
        \item 図:跨いだReIDの図(全スライドと対応するように)
        \item 図:深さ方向に追跡を行うことで,どれが同じ個体由来であるかを知ることができる
    \end{itemize}
\end{frame}

\begin{frame}[noframenumbering]{提案手法2:三次元位置の状態予測}
    \begin{itemize}
        \item 口頭:
        \begin{itemize}
            \item 次に,スライスを跨いだ複数のバウンディングボックスから推定をする
            \item カルマンフィルタによって行う
            \item 貢献として,EKFと可変的な観測変数を用いていることもいう
            \item or カルマンフィルタの導入から話す
        \end{itemize}
        \item 図:行っていることを図示化?
    \end{itemize}
\end{frame}

