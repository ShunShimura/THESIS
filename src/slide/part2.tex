\section{問題設定}

% \begin{frame}[noframenumbering]{AIシステムによるリアルタイム細胞追跡}
%     \begin{itemize}
%         \item 口頭
%         \begin{itemize}
%             \item 前述のロボットによる自動分取,位置情報の保存をするために追跡
%             \item 図のように,各細胞がどこにいるのかを画像から認識
%             \item 最新の位置をロボットに送り,最初の位置が空間的な情報
%         \end{itemize}
%         \item 図:オンライン処理,最初と最新の位置に赤丸,ここでは単一画像
%     \end{itemize}
% \end{frame}
\begin{frame}{AIシステムによるリアルタイム細胞追跡}
    \begin{itemize}
        \item AIシステム:すべての細胞をリアルタイム追跡
        \\\ra \red{\dotuline{\black{初期位置}}}を保持しながら\red{\uline{\black{最新位置}}}をロボットに送信
    \end{itemize}
    \myfigure{slide/slide3.pdf}
\end{frame}

% \begin{frame}[noframenumbering]{共焦点顕微鏡から得られる四次元データ}
%     \begin{itemize}
%         \item 口頭
%         \begin{itemize}
%             \item 入力画像について詳しく
%             \item 三次元空間を取りたい(すべての細胞,縦向きの動き)
%             \item 光学的な断面像を取れる共焦点顕微鏡を使う
%             \item 順番に高さを変えながら,設定したところまでいったら戻る,を繰り返す
%             \item 二次元+高さ+時間の四次元データが得られる
%             \item 二つの分解能を上げれば,簡単に目的を達成できる
%             \item しかし,有限の時間がかかってしまう
%         \end{itemize}
%         \item 図:得られる画像の時系列,時間の数値を書く
%     \end{itemize}
% \end{frame}
\begin{frame}{共焦点顕微鏡から得られる四次元データ}
    \begin{itemize}
        \item 細胞は三次元空間を遊離する($\because$ 酵素溶液処理)
        \item リアルタイムに細胞を分取したい($\because$ 細胞が移動する)
    \end{itemize}
    \ra 共焦点顕微鏡によって,\red{スライスした複数の二次元画像}で三次元空間を認識
    \\\phantom{\ra しかしながらこの共焦点画像の取得には,\blue{1秒/1枚かかる}}
    \myfigure{slide/slide4-1.pdf}
\end{frame}
\begin{frame}[noframenumbering]{共焦点顕微鏡から得られる四次元データ}
    \begin{itemize}
        \item 細胞は三次元空間を遊離する($\because$ 酵素溶液処理)
        \item リアルタイムに細胞を分取したい($\because$ 細胞が移動する)
    \end{itemize}
    \ra 共焦点顕微鏡によって,\red{スライスした複数の二次元画像}で三次元空間を認識
    \\\ra しかしながらこの共焦点画像の取得には,\blue{1秒/1枚かかる}
    \myfigure{slide/slide4-2.pdf}
\end{frame}

% \begin{frame}[noframenumbering]{細胞分取のための三次元位置予測}
%     \begin{itemize}
%         \item 口頭
%         \begin{itemize}
%             \item 共焦点顕微鏡入力では,時間的,空間的に疎になる
%             \item 細胞の動きがあったり,スライスに完全に映らなかったり
%             \item 分取するためには,この情報を復元かつ予測しなければならない
%             \item 結論として,追跡しながら高精度な三次元位置の推定が必要,これが出力
%         \end{itemize}
%         \item 図:二点目を表す図
%     \end{itemize}
% \end{frame}
\begin{frame}{AIシステムによる細胞位置の予測}
    \begin{itemize}
        \item \uline{空間分解能の必要性}
        \begin{itemize}
            \item ロボットによる取得のために,高精度に位置を予測したい
            \item 高さを推定するには,細かいスライス間隔が必要
        \end{itemize}
        \vspace{0.5zh}
        \item \uline{時間分解能の必要性}
        \begin{itemize}
            \item すべてのスライスを同時に確認できない
            \item 動く細胞を追従しなければならない
        \end{itemize}
    \end{itemize}
    \vs
    \begin{block}{AIシステムのタスク}
        \begin{itemize}
            \item 入力:時間方向および高さ方向に疎な画像時系列データ
            \item 処理:逐次的にすべての遊離する細胞を追跡
            \item 出力:一定時間後の細胞の位置予測
        \end{itemize}
    \end{block}
\end{frame}