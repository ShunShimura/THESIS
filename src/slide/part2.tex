\section{問題設定}

\begin{frame}[noframenumbering]{AIシステムによるリアルタイム細胞追跡}
    \begin{itemize}
        \item 口頭
        \begin{itemize}
            \item 前述のロボットによる自動分取,位置情報の保存をするために追跡
            \item 図のように,各細胞がどこにいるのかを画像から認識
            \item 最新の位置をロボットに送り,最初の位置が空間的な情報
        \end{itemize}
        \item 図:オンライン処理,最初と最新の位置に赤丸,ここでは単一画像
    \end{itemize}
\end{frame}
\begin{frame}{AIシステムによるリアルタイム細胞追跡}
    \myfigure{slide/slide3.pdf}
\end{frame}

\begin{frame}[noframenumbering]{共焦点顕微鏡から得られる四次元データ}
    \begin{itemize}
        \item 口頭
        \begin{itemize}
            \item 入力画像について詳しく
            \item 三次元空間を取りたい(すべての細胞,縦向きの動き)
            \item 光学的な断面像を取れる共焦点顕微鏡を使う
            \item 順番に高さを変えながら,設定したところまでいったら戻る,を繰り返す
            \item 二次元+高さ+時間の四次元データが得られる
            \item 二つの分解能を上げれば,簡単に目的を達成できる
            \item しかし,有限の時間がかかってしまう
        \end{itemize}
        \item 図:得られる画像の時系列,時間の数値を書く
    \end{itemize}
\end{frame}
\begin{frame}{共焦点顕微鏡から得られる四次元データ}
    \myfigure{slide/slide4.pdf}
\end{frame}

\begin{frame}[noframenumbering]{細胞分取のための三次元位置予測}
    \begin{itemize}
        \item 口頭
        \begin{itemize}
            \item 共焦点顕微鏡入力では,時間的,空間的に疎になる
            \item 細胞の動きがあったり,スライスに完全に映らなかったり
            \item 分取するためには,この情報を復元かつ予測しなければならない
            \item 結論として,追跡しながら高精度な三次元位置の推定が必要,これが出力
        \end{itemize}
        \item 図:二点目を表す図
    \end{itemize}
\end{frame}