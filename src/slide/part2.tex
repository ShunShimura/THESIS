\section{問題設定}

\begin{frame}[noframenumbering]{目次}
    \tableofcontents[currentsection]
\end{frame}

\begin{frame}{AIシステムによるリアルタイム複数細胞追跡}
    \begin{itemize}
        \item AIシステム:\red{すべての細胞をリアルタイム追跡}
        \\\ra \red{\dotuline{\black{元住所}}}を保持しながら\red{\uline{\black{最新位置}}}をロボットに送信
    \end{itemize}
    \myfigure{slide/slide3-0.pdf}
\end{frame}
\begin{frame}[noframenumbering]{AIシステムによるリアルタイム複数細胞追跡}
    \begin{itemize}
        \item AIシステム:\red{すべての細胞をリアルタイム追跡}
        \\\ra \red{\dotuline{\black{元住所}}}を保持しながら\red{\uline{\black{最新位置}}}をロボットに送信
    \end{itemize}
    \myfigure{slide/slide3-1.pdf}
\end{frame}
\begin{frame}[noframenumbering]{AIシステムによるリアルタイム複数細胞追跡}
    \begin{itemize}
        \item AIシステム:\red{すべての細胞をリアルタイム追跡}
        \\\ra \red{\dotuline{\black{元住所}}}を保持しながら\red{\uline{\black{最新位置}}}をロボットに送信
    \end{itemize}
    \myfigure{slide/slide3-2.pdf}
\end{frame}
\begin{frame}[noframenumbering]{AIシステムによるリアルタイム複数細胞追跡}
    \begin{itemize}
        \item AIシステム:\red{すべての細胞をリアルタイム追跡}
        \\\ra \red{\dotuline{\black{元住所}}}を保持しながら\red{\uline{\black{最新位置}}}をロボットに送信
    \end{itemize}
    \myfigure{slide/slide3-3.pdf}
\end{frame}
\begin{frame}[noframenumbering]{AIシステムによるリアルタイム複数細胞追跡}
    \begin{itemize}
        \item AIシステム:\red{すべての細胞をリアルタイム追跡}
        \\\ra \red{\dotuline{\black{元住所}}}を保持しながら\red{\uline{\black{最新位置}}}をロボットに送信
    \end{itemize}
    \myfigure{slide/slide3-4.pdf}
\end{frame}

\begin{frame}{共焦点顕微鏡から得られる4Dデータ}
    \begin{itemize}
        \item 時系列データが必要($\because$ 移動する細胞を取得する)
        \item 三次元データが必要($\because$ 酵素溶液処理により流体に満たされている)
    \end{itemize}
    \ra 共焦点顕微鏡\cite{paddock2000principles}により,\red{複数の高さでスライスした二次元画像の時系列データ}を取得
    \\\phantom{\ra \blue{\textbf{Challenging}}:画像の取得に,\blue{1秒/1枚}かかる}
    \myfigure{slide/slide4-1.pdf}
\end{frame}
\begin{frame}[noframenumbering]{共焦点顕微鏡から得られる4Dデータ}
    \begin{itemize}
        \item 時系列データが必要($\because$ 移動する細胞を取得する)
        \item 三次元データが必要($\because$ 酵素溶液処理により流体に満たされている)
    \end{itemize}
    \ra 共焦点顕微鏡\cite{paddock2000principles}により,\red{複数の高さでスライスした二次元画像の時系列データ}を取得
    \\\ra \blue{\textbf{Challenging}}:画像の取得に,\blue{1秒/1枚}かかる
    \myfigure{slide/slide4-2.pdf}
\end{frame}

\begin{frame}{AIシステムによる見えない細胞の位置予測}
    \begin{itemize}
        \item 高い時間分解能が必要
        \begin{itemize}
            \item 動く細胞を追従して細胞を追跡しなければならない
            \item 元住所を保持するために,FPSは上げておきたい
        \end{itemize}
        \vs
        \item 高い空間分解能が必要
        \begin{itemize}
            \item ロボットによる取得のために,高精度に位置を知りたい
            \item できる限りすべての細胞を見逃したくない
        \end{itemize}
    \end{itemize}
    \Ra トレードオフにより,\red{見えていない細胞を取りに行く必要}が生じる
    \vs
    \begin{block}{AIシステムの有効性}
        \begin{enumerate}
            \item 時間的・空間的に疎なデータから,\red{追跡および高精度な位置予測}を行う
            \item 上記の処理を\red{高速}に行う
        \end{enumerate}
    \end{block}
\end{frame}