\section{計算機実験}

\begin{frame}[noframenumbering]{目次}
    \tableofcontents[currentsection]
\end{frame}

\begin{frame}{検証用データによる状態予測評価}
    \begin{itemize}
        \item 検証用データを用いて,\red{三次元位置の予測精度}を評価
    \end{itemize}
    \vspace{0.5zh}
    \uline{評価(1)予測位置および正解位置の可視化}
    \begin{columns}
        \begin{column}{.6\linewidth}
            % \begin{figure}
            %     \animategraphics[autoplay, loop, width=\linewidth]{20}{fig/slide/3d_spheres/t}{001}{060}
            % \end{figure}              
            \myfigure[.8]{slide/3d_spheres/t030.pdf}
        \end{column}
        \begin{column}{.4\linewidth}
            \begin{itemize}
                \item[$\blacktriangleright$] 左側(\red{赤})が予測された位置
                \item[$\blacktriangleright$] 右側(\blue{青})が正解の位置
            \end{itemize}
        \end{column}
    \end{columns}
    \uline{評価(2)予測精度の定量化}
    \begin{columns}
        \begin{column}{.6\linewidth}
            \myfigure[0.9]{slide/11.pdf}
        \end{column}
        \begin{column}{.4\linewidth}
            \vspace{-2.5zh}
            \begin{itemize}
                \item[$\blacktriangleright$] 横軸は時刻を表すステップ
                \item[$\blacktriangleright$] 予測と正解の\uline{IoU}を計算
                \item[$\blacktriangleright$] 常にほとんど$1.0$(\red{good})
            \end{itemize}
        \end{column}
    \end{columns}
\end{frame}

\begin{frame}{遊離した細胞の自動分取}
    
\end{frame}
