\section{研究背景}

\begin{frame}[noframenumbering]{目次}
    \tableofcontents[currentsection]
\end{frame}

\begin{frame}{植物の再生メカニズム解明}
    \begin{itemize}
        \item \red{植物のリプログラミング}:たった一つの細胞からでも個体にまで再生
        \\\ra 人工的に再生を誘発 \Ra 農作物の生産率向上,絶滅危惧種の保全
    \end{itemize}
    \vs
    \uline{single cell RNA-seq}\ \cite{kolodziejczyk2015technology}:\red{どこの}細胞が\red{どのような}働きをしたのか
    \myfigure[0.8]{slide/slide1.pdf}
    \Ra いまだ,\red{一細胞レベルの空間分解能を持った}scRNA-seqは行われていない\footnote{タイプごと(道管細胞や根冠細胞など)に仕分けたscRNA-seqであれば報告されている\cite{shahan2022single}}
    \vs
\end{frame}

\begin{frame}{位置情報付き一細胞分取システム}
    \begin{block}{Advantages}
        \begin{itemize}
            \item 細胞分取の自動化 \Ra 統計的検出力の確保
            \item 元住所の保持 \Ra \red{一細胞レベルの空間分解能}
        \end{itemize}
    \end{block}
    \myfigure[0.8]{slide/slide2-1.pdf}
\end{frame}
\begin{frame}[noframenumbering]{位置情報付き一細胞分取システム}
    \begin{block}{Advantages}
        \begin{itemize}
            \item 細胞分取の自動化 \Ra 統計的検出力の確保
            \item 元住所の保持 \Ra \red{一細胞レベルの空間分解能}
        \end{itemize}
    \end{block}
    \myfigure[0.8]{slide/slide2-2.pdf}
\end{frame}