\section{研究背景}

% \begin{frame}[noframenumbering]{植物の再生メカニズムの解明}
%     \begin{itemize}
%         \item 口頭:再生力の強さ,社会的貢献,scRNA-seqの目的
%         \item 図:どこの細胞がどのような働きをしたのか(ナズナの論文引用?)
%         \item つなぎ:部分的であり,一細胞レベル分解能はまだ
%     \end{itemize}
% \end{frame}
\begin{frame}{植物の再生メカニズムの解明}
    \begin{itemize}
        \item 植物のリプログラミング:たった一つの細胞からでも個体にまで再生
        \\\ra scRNA-seq:\red{どこの}細胞が,\red{どのような}働きをしたのかを解析
    \end{itemize}
    \begin{figure}
        \centering
        \includegraphics[width=0.7\linewidth]{fig/slide/slide1.pdf}
        \caption{シロイヌナズナのscRNA-seq結果(\cite{shahan2022single}より引用)}
    \end{figure}
    \Ra いまだ,\red{一細胞レベルの空間分解能}をもったscRNA-seqは行われていない
\end{frame}

% \begin{frame}[noframenumbering]{位置情報付き一細胞分取システム}
%     \begin{itemize}
%         \item 口頭:システムの構築,一細胞ごと収集,統計的推定のために自動化,保持する
%         \item 図:画像,AIシステム,ロボット,遊離,コマ送りでAI
%         \item 貢献も述べる??
%     \end{itemize}
% \end{frame}
\begin{frame}{位置情報付き一細胞分取システム}
    \begin{block}{システムの利点}
        \begin{itemize}
            \item AIとロボットにより,細胞分取を自動化 \Ra 統計的検出力の確保
            \item 各細胞の元住所を保持したまま細胞を分取 \Ra \red{一細胞レベルの空間分解能}
        \end{itemize}
    \end{block}
    \vspace{-1zh}
    \myfigure[0.8]{slide/slide2-1.pdf}
\end{frame}
\begin{frame}[noframenumbering]{位置情報付き一細胞分取システム}
    \begin{block}{システムの利点}
        \begin{itemize}
            \item AIとロボットにより,細胞分取を自動化 \Ra 統計的検出力の確保
            \item 各細胞の元住所を保持したまま細胞を分取 \Ra \red{一細胞レベルの空間分解能}
        \end{itemize}
    \end{block}
    \vspace{-1zh}
    \myfigure[0.8]{slide/slide2-2.pdf}
\end{frame}