この章では,個性付き1細胞取得システムを達成するためのAIシステムの概要説明,およびAIが解くタスクの定式化を行う.

\section{AIシステムの概要}
上述の通り,本プロジェクトでは細胞の個性として遊離前の細胞の位置情報を設定している.その位置情報を保持するために,AIを用いてリアルタイム細胞追跡を行う.これにより得られる効果は以下の2つである.
\begin{itemize}
    \item 各細胞の現在の位置,及び一定時間後の位置を予測することができる.
    \item 各細胞に対して,遊離する前のそれぞれの位置を保持することができる.
\end{itemize}
また,AIに入力するデータは,共焦点顕微鏡から得られる画像データである.ここで共焦点顕微鏡とは,焦点と共役する位置にピンホールを置くことで焦点外で発生した光を排除し,対象物体の光学断面像(以降スライスと呼ぶ)を得ることができる顕微鏡である(図).
