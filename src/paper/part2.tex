\thispagestyle{fancy2}

\section{AIシステムの役割}
個性付き一細胞分取システムは,プロトプラスト化した細胞を一つずつ収集し,かつそれら細胞すべてについてプロトプラスト化する以前の根における領域を特定すること
% (図\ref{fig:system}参照)
を目的とし,AI・ロボットによるそれらの完全自動化を目指していた.また自動化においては,一定時間内で収集できる細胞数を増やしたり,収集の失敗をなるべく削減したりなど,効率性及び正確性も求められる.

% \begin{figure}[t]
%     \centering
%     \includegraphics[width=.3\textwidth]{fig/nagoya-u.eps}
%     \caption{個性付き一細胞分取システムの目的}
%     \label{fig:system}
% \end{figure}

\par
本システムの主な要素は,リアルタイムでプロトプラスト化の様子を撮影する顕微鏡,プロトプラスト化した細胞の収集を行うロボット,その他目的を達成するためのAIの3つである.このAIが求められているタスクは以下の2つである.

\begin{enumerate}[label=(\arabic*)]
    \item ロボットが向かうべき位置を示すこと.
    \item 取得した細胞のプロトプラスト化以前の領域を特定すること.
\end{enumerate}

これらのタスクを達成するために,AIシステムではリアルタイム細胞追跡を行う.より具体的には,顕微鏡から得られた画像をリアルタイムで処理し,各細胞の位置を認識,さらに各細胞にIDを割り振ることで追跡を行う.まず(1)のタスクについては,追跡している各細胞の最新の位置か若しくは追跡した軌跡から推定した予測位置を出力し,何らかの方策最適化で取りに行く細胞を一つ選べば良い.この方策最適化については本論文の範囲外のため割愛させていただく.(2)のタスクについては,追跡してきた軌跡を逆向きに辿れば達成できる.
\par
以降の節では,AIシステムの入力及び出力について詳細に述べる.

\subsection{共焦点顕微鏡から得られる4Dデータ}
本システムでは,植物の根及びプロトプラスト化した細胞は酵素溶液中,すなわち流体環境にさらされているため,3次元空間を遊離する.そのため細胞の追跡を行うためには,3次元空間を認識する顕微鏡を使用しなければならない.
\par
共焦点顕微鏡は,このような問題に対処することができる.共焦点顕微鏡とは,焦点と共役する位置にピンホールを置くことで焦点外で発生した光を排除し,対象物体の光学断面像(以降スライスと呼ぶ)を得ることができる顕微鏡である(図\ref{fig:confocal}参照).焦点距離は可変であるため,スライスの高さを変えながら撮影することができる.すなわち3次元空間を,複数の高さでスライスした2次元画像の集合,という形式で捉えることが可能である.本システムでは,この焦点面の高さを変えながら2次元スライス画像を取得しつつ,時間方向にこれを繰り返すことで,空間3次元+時間の4次元データを取得する.ただしここでは,焦点面の高さが連続する2枚の画像間でも有限の時間差が生じることに注意されたい.

\begin{figure}[t]
    \centering
    \includegraphics[width=\textwidth]{fig/confocal.pdf}
    \caption{共焦点顕微鏡の概略図}
    \small
    左の図では緑色の光が焦点で集まり,うまく検出されている.これにより,緑色の光が集まっている点の高さの像を捉えられる.一方右の図の赤色の光が集まっている点の高さでは,光は検出器手前のピンホールに遮られ,排除される.
    \label{fig:confocal}
\end{figure}

\subsection{遊離した細胞の位置予測}
プロトプラスト化して遊離した細胞が,ほとんど運動をしていないか静止している場合には,最新の画像から得られた位置にロボットを向かわせれば十分に取得できる.しかしながら当然遊離した細胞は速度を持ち,それに加えてロボットを動かすことによる流体外乱によってその速度は影響を受ける.また前述のとおり,画像は細胞の運動に関係なく撮影されるため,最新の画像から得られた位置に向かわせるだけでは不十分である.
\par
そこで本システムでは,AIによる細胞の位置予測を行う.より具体的には,一定時間後の予測位置のみを出力するのではなく,速度や加速度も加えて予測する.これにより,ロボットがある細胞の位置へ到着するまでにかかる時間やその経路も考慮に入れて細胞を収集することができる.

\subsection{遊離前領域との対応付け}
遊離する前の領域を特定するためには,図\ref{fig:segmentation}に示すようなセグメンテーションという処理が適している.そのためすべての時刻に対してセグメンテーションを実施し,細胞追跡を行うのが直感的な手法である.
しかしながら一般的に,セグメンテーションの推論時間には数分以上を要してしまうため,リアルタイムな処理に適していない.
\par
そこで本システムでは,遊離前の領域特定はオフラインなセグメンテーションを行い,遊離直後からの細胞追跡はオンラインな物体検出を行う.そしてオフライン処理時に,図\ref{fig:segmentation_matching}のように遊離直後の位置情報と遊離前のセグメンテーションのID対応付けを行う.
\par
よって細胞追跡では,遊離する前の細胞は位置認識せず,プロトプラスト化して遊離している細胞のみを追跡対象とする.対応付けの手法については本論文の範囲外のため,割愛させていただく.

\begin{figure}[t]
    \centering
    \begin{subfigure}[b]{.4\textwidth}
        \centering
        \includegraphics[width=\textwidth]{fig/segmentation.pdf}
        \caption{セグメンテーションの例}
        \label{fig:segmentation}
    \end{subfigure}
    \hspace{30pt}
    \begin{subfigure}[b]{.4\textwidth}
        \centering
        \includegraphics[width=\textwidth]{fig/segmentation_matching.pdf}
        \caption{遊離直後位置とのマッチング}
        \label{fig:segmentation_matching}
    \end{subfigure}
    \small \\
    (a)厳密にはインスタンスセグメンテーションと呼ばれる,個々の物体ごとに領域を検出する処理である.この例では車のみを対象としている.(b)セグメンテーションの結果と追跡によって得た遊離直後の位置のマッチング.図では橙色と水色の細胞がそれぞれ同一細胞である.
\end{figure}

\section{問題設定の一般化及び定式化}



\section{AIシステムの課題点}