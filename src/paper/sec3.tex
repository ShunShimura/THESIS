\section{既存手法と問題点}

複数物体追跡は,単一画像の時系列データを入力し,画像内に映るある物体が他の時刻ではどこに存在するのかを特定するタスクである.一般的な複数物体追跡手法では,画像における各物体$i$の位置$\bm{o}^{(i)}$を中心位置$x,y$と幅$w$,高さ$h$を用いて,バウンディングボックスと呼ばれる
\begin{equation}
    \notag
    \bm{o}^{(i)} = [x, y, w, h]
\end{equation}
で表す.そして追跡の段階では,このバウンディングボックスにIDが振られることで目的が達成される.

しかしながらこの手法を本研究に適用するためには,二つの問題点が生じる.
\begin{description}
    \item[状態推定問題] 上記の手法では,物体の位置を二次元的な情報でしか得られず,三次元的な細胞の位置を正確に予測することが困難である.
    \item[マルチスライス問題] 三次元的な情報を得るには,他の焦点面におけるバウンディングボックスに同じIDを振る必要があるが,上記の既存手法では,単一の焦点面のバウンディングボックスにしか共通のIDを降ることができない.
\end{description}