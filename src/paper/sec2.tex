
われわれは流体にさらされた三次元空間中の細胞を追跡するために,共焦点顕微鏡から得られる複数の焦点面で得られた二次元画像群の時系列データを取得する.
共焦点顕微鏡は焦点外の光を除去することで,光学的な断面像(以降スライスと呼ぶ)を取得することができる.
この焦点面を変えながらいくつか撮影することで,三次元空間情報を捉えた複数の二次元画像を得ることができる.
そしてこの複数の二次元画像時系列データを入力し,各時刻における細胞の三次元的な位置を推定・追跡する.