\section{提案手法}

上記の状態推定問題を解決するために,複数の焦点面におけるバウンディングボックス群
\begin{equation}
    \notag
    B^{(i)} = \left\{\bm{b}_{t,s}^{(i)} \mid t = 1,\dots, \mid s = 1, \dots,\right\}
\end{equation}
から細胞$i$の状態変数$\bm{s}_t^{(i)}$を推定するカルマンフィルタ,SliceKalmanFilterを提案する.ここで$t$は時刻,$s$はスライスを表す.

また時刻$t$において複数の二次元画像から検出されたバウンディングボックス群を各個体由来のものに仕分けるためのDepthSORTという手法を提案する.このDepthSORTでは焦点面の高さ方向に追跡を行うことで,同じ個体由来だと考えられる他の焦点面のバウンディングボックスに共通のIDを振り分ける.

\section{研究成果}

図\ref{fig:meanIoU}に結果を示す.縦軸は時刻$t-1$までの情報から時刻$t$の細胞の位置を予測したときの正解ラベルとの重なり具合である.また横軸はその時刻を表す.

\begin{figure}[h]
    \centering
    \includegraphics[width=\linewidth]{fig/meanIoUs.pdf}
    \label{fig:meanIoU}
\end{figure}
