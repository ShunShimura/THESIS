\section{課題点と提案手法}

既存のMOT手法は,単一の二次元画像時系列データを入力として想定している.そのため本問題設定に適用するには,各スライスを単一の二次元画像時系列データとして入力する方法が考えられる.

しかしながら本問題設定において,各スライスごとに既存のMOT手法を適用するだけで正確に三次元的な位置の推定を達成することは難しい.実際,三次元的な位置の推定を行うためには,各スライスの情報を何かしらの方法で統合する必要がある.

そこで提案手法として,まずスライスを跨いで同一個体由来のバウンディングボックスに同じIDを振るためのDepthSORTという処理を提案する.そして次に,複数のスライスにおけるバウンディングボックス情報から三次元的な位置を推定するためのカルマンフィルタ,SliceKalmanFilterを提案する.

\section{結果}

図\ref{fig:meanIoUs}に検証データを用いた,既存手法と提案手法の比較結果を示す.ここで追跡の精度を評価するために,時刻$t-1$までの時系列データを入力し,時刻$t$の三次元的な位置を推定したときの,正解ラベル(真の細胞の位置)との重なり具合を示すmeanIoUという評価指標を導入し,縦軸にプロットした.この評価指標は,推定値と正解ラベルが完全に一致した場合には$1.0$になる.また横軸のStepはその時刻$t$を表している.図に示すように,すべての時刻において提案手法の推定精度が既存手法を大きく上回っていることが分かる.

\begin{figure}[h]
    \centering
    \includegraphics[width=\linewidth]{fig/meanIoUs4abstract.pdf}
    \caption{既存手法と提案手法の計算機実験比較}
    \label{fig:meanIoUs}
\end{figure}
